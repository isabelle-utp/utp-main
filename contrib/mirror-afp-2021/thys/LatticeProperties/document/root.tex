\documentclass[11pt,a4paper]{article}
\usepackage{isabelle,isabellesym}
\usepackage{amssymb}
\usepackage[utf8]{inputenc}
\usepackage[only,bigsqcap]{stmaryrd}

% this should be the last package used
\usepackage{pdfsetup}

% urls in roman style, theory text in math-similar italics
\urlstyle{rm}
\isabellestyle{it}


\begin{document}

\title{Lattice Properties}

\author{Viorel Preoteasa}

\maketitle

\begin{abstract}
This formalization introduces and collects some algebraic
structures based on lattices and complete lattices for use
in other developments. The structures introduced are modular, 
and lattice ordered groups. In addition to the results proved for 
the new lattices, this formalization also introduces theorems 
about latices and complete lattices in general.
\end{abstract}

\tableofcontents

\parindent 0pt\parskip 0.5ex

\section{Overview}

Section 2 introduces well founded and transitive relations.
Section 3 introduces some properties about fixpoints of
monotonic application which maps monotonic functions to
monotonic functions.  The most important property is that 
such a monotonic application has the least fixpoint monotonic.
Section 4 introduces conjunctive, disjunctive, universally
conjunctive, and universally disjunctive functions. 
In section 5 some simplification lemmas for alttices are proved.
Section 6 introduces modular lattices and proves some properties
about them and about distributive lattices. The main result
of this section is that a lattice is distributive if and only
if it satisfies
$$\forall x \; y \; z: x \sqcap z = y \sqcap z \land  x \sqcup z = y \sqcup z \longrightarrow x = y$$
Section 7 introduces lattice ordered groups and some of their properties.
The most important is that they are distributive lattices, and
this property is proved using the results from Section 5.

% generated text of all theories
\input{session}

% optional bibliography
\bibliographystyle{abbrv}
\bibliography{root}

\end{document}

%%% Local Variables:
%%% mode: latex
%%% TeX-master: t
%%% End:
