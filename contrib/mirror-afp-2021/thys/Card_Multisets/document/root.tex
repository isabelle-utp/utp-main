\documentclass[11pt,a4paper]{article}
\usepackage{isabelle,isabellesym}

% this should be the last package used
\usepackage{pdfsetup}

% urls in roman style, theory text in math-similar italics
\urlstyle{rm}
\isabellestyle{it}

\begin{document}

\title{Cardinality of Multisets}
\author{Lukas Bulwahn}
\maketitle

\begin{abstract}

This entry provides three lemmas to count the number of multisets of a
given size and finite carrier set.
The first lemma provides a cardinality formula assuming that the
multiset's elements are chosen from the given carrier set. 
The latter two lemmas provide formulas assuming that the multiset's
elements also cover the given carrier set, i.e., each element of the carrier
set occurs in the multiset at least once.

The proof of the first lemma uses the argument of the recurrence relation for
counting multisets~\cite{wikipedia:Multiset}. The proof of the second lemma is
straightforward, and the proof of the third lemma is easily obtained
using the first cardinality lemma.  
A challenge for the formalization is the derivation of the required
induction rule, which is a special combination of the induction
rules for finite sets and natural numbers. The induction rule is derived
by defining a suitable inductive predicate and transforming the predicate's
induction rule.

\end{abstract}

\tableofcontents

% sane default for proof documents
\parindent 0pt\parskip 0.5ex

% generated text of all theories
\input{session}

\nocite{*}

\bibliographystyle{abbrv}
\bibliography{root}

\end{document}

%%% Local Variables:
%%% mode: latex
%%% TeX-master: t
%%% End:
