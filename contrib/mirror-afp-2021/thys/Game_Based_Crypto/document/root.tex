\documentclass[11pt,a4paper]{article}
\usepackage{isabelle,isabellesym}
\usepackage{amssymb,amsmath}
\usepackage[english]{babel}
\usepackage[only,bigsqcap]{stmaryrd}
\usepackage{wasysym}
\usepackage{booktabs}
\usepackage{authblk}
\usepackage[inline]{enumitem}
\usepackage{amsthm}
\usepackage{mathptmx}
\usepackage{tikz}
\usetikzlibrary{%
  arrows,%
  arrows.meta,%
  calc,%
  chains,%
  patterns,%
  decorations.pathreplacing,%
  fit,%
  intersections,%
  positioning,%
  shapes.multipart,%
  svg.path,%
}


% this should be the last package used
\usepackage{pdfsetup}

% urls in roman style, theory text in math-similar italics
\urlstyle{rm}
\isabellestyle{it}

\newcommand{\CryptHOL}{CryptHOL}
%
\theoremstyle{definition}
\newtheorem*{definition}{Definition}

\begin{document}

\title{Game-based cryptography in HOL}
\author{Andreas Lochbihler and S. Reza Sefidgar and Bhargav Bhatt}
\maketitle

\begin{abstract}
  In this AFP entry, we show how to specify game-based cryptograph\-ic security notions and formally prove secure
  several cryptographic constructions from the literature using the CryptHOL framework.
  Among others, we formalise the notions of a random oracle, a pseudo-random function, an
  unpredictable function, and of encryption schemes that are indistinguishable under chosen plaintext
  and/or ciphertext attacks.
  We prove the random-permutation/random-function switching lemma, security of the Elgamal
  and hashed Elgamal public-key encryption scheme and correctness and security of several
  constructions with pseu\-do-random functions.

  Our proofs follow the game-hopping style advocated by Shoup \cite{Shoup2004IACR} and Bellare and
  Rogaway \cite{BellareRogaway2006EUROCRYPT}, from which most of the examples have been taken.
  We generalise some of their results such that they can be reused in other proofs.
  Thanks to CryptHOL's integration with Isabelle's parametricity infrastructure, many simple hops
  are easily justified using the theory of representation independence.
\end{abstract}


\tableofcontents

\clearpage

% sane default for proof documents
\parindent 0pt\parskip 0.5ex

% generated text of all theories
\input{session}

% optional bibliography
\bibliographystyle{abbrv}
\bibliography{root}

\end{document}

%%% Local Variables:
%%% mode: latex
%%% TeX-master: t
%%% End:
