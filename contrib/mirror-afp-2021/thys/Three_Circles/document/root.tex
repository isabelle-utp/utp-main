\documentclass[11pt,a4paper]{article}
\usepackage{isabelle,isabellesym}

% further packages required for unusual symbols (see also
% isabellesym.sty), use only when needed

%\usepackage{amssymb}
  %for \<leadsto>, \<box>, \<diamond>, \<sqsupset>, \<mho>, \<Join>,
  %\<lhd>, \<lesssim>, \<greatersim>, \<lessapprox>, \<greaterapprox>,
  %\<triangleq>, \<yen>, \<lozenge>

%\usepackage{eurosym}
  %for \<euro>

%\usepackage[only,bigsqcap]{stmaryrd}
  %for \<Sqinter>

%\usepackage{eufrak}
  %for \<AA> ... \<ZZ>, \<aa> ... \<zz> (also included in amssymb)

%\usepackage{textcomp}
  %for \<onequarter>, \<onehalf>, \<threequarters>, \<degree>, \<cent>,
  %\<currency>

% this should be the last package used
\usepackage{pdfsetup}

% urls in roman style, theory text in math-similar italics
\urlstyle{rm}
\isabellestyle{it}

% for uniform font size
%\renewcommand{\isastyle}{\isastyleminor}


\begin{document}

\title{The Theorem of Three Circles}
\author{Fox Thomson, Wenda Li}
\maketitle


\begin{abstract}
  The Descartes test based on Bernstein coefficients and Descartes' rule of signs 
  effectively (over-)approximates the number of real roots of a univariate polynomial over 
  an interval. In this entry we formalise the theorem of three circles (Theorem 10.50 in \cite{Basu:2016bo}), 
	which gives sufficient conditions for when the Descartes test 
	returns 0 or 1. This is the first step for efficient root isolation.
\end{abstract}

\tableofcontents


% sane default for proof documents
\parindent 0pt\parskip 0.5ex

% generated text of all theories
\input{session}


\section{Acknowledgements}

The work has been jointly supported by the Cambridge Mathematics Placements (CMP) 
Programme and the ERC Advanced Grant ALEXANDRIA (Project GA 742178).

% optional bibliography
\bibliographystyle{abbrv}
\bibliography{root}

\end{document}

%%% Local Variables:
%%% mode: latex
%%% TeX-master: t
%%% End:
