\documentclass[11pt,a4paper]{article}
\usepackage{isabelle,isabellesym}
\usepackage[english]{babel}  % for guillemots

% this should be the last package used
\usepackage{pdfsetup}

% urls in roman style, theory text in math-similar italics
\urlstyle{rm}
\isabellestyle{it}

\begin{document}

\title{The Nash-Williams Theorem}
\author{Lawrence C. Paulson}
\maketitle

\begin{abstract}
In 1965, Nash-Williams~\cite{nash-williams-quasi} discovered a generalisation of the infinite form of Ramsey's theorem. Where the latter concerns infinite sets of $n$-element sets for some fixed~$n$, the Nash-Williams theorem concerns infinite sets of finite sets (or lists) subject to a ``no initial segment'' condition. The present formalisation follows Todor\v{c}evi{\'c} \cite{todorcevic-ramsey}.
\end{abstract}

\tableofcontents

% include generated text of all theories
\input{session}

\section{Acknowledgements}
The author was supported by the ERC Advanced Grant ALEXANDRIA (Project 742178) funded by the European Research Council. Todor\v{c}evi{\'c} provided help with the proofs by email.

\bibliographystyle{abbrv}
\bibliography{root}

\end{document}
