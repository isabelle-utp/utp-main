\documentclass[11pt,a4paper]{article}
\usepackage{isabelle,isabellesym}
\usepackage{amsfonts, amsmath, amssymb}

% this should be the last package used
\usepackage{pdfsetup}

% urls in roman style, theory text in math-similar italics
\urlstyle{rm}
\isabellestyle{it}


\begin{document}

\title{The Transcendence of $\pi$}
\author{Manuel Eberl}
\maketitle

\begin{abstract}
This entry shows the transcendence of $\pi$ based on the classic proof using the fundamental theorem of symmetric polynomials first given by von Lindemann in 1882, but the mostly formalisation follows the version by Niven~\cite{niven_pi39}. The proof reuses much of the machinery developed in the AFP entry on the transcendence of $e$.
\end{abstract}

\tableofcontents
\newpage
\parindent 0pt\parskip 0.5ex

\input{session}

\bibliographystyle{abbrv}
\bibliography{root}

\end{document}

%%% Local Variables:
%%% mode: latex
%%% TeX-master: t
%%% End:
