\documentclass[11pt,a4paper]{article}
\usepackage{isabelle,isabellesym}

% further packages required for unusual symbols (see also
% isabellesym.sty), use only when needed

%\usepackage{amssymb}
  %for \<leadsto>, \<box>, \<diamond>, \<sqsupset>, \<mho>, \<Join>,
  %\<lhd>, \<lesssim>, \<greatersim>, \<lessapprox>, \<greaterapprox>,
  %\<triangleq>, \<yen>, \<lozenge>

%\usepackage{eurosym}
  %for \<euro>

%\usepackage[only,bigsqcap]{stmaryrd}
  %for \<Sqinter>

%\usepackage{eufrak}
  %for \<AA> ... \<ZZ>, \<aa> ... \<zz> (also included in amssymb)

%\usepackage{textcomp}
  %for \<onequarter>, \<onehalf>, \<threequarters>, \<degree>, \<cent>,
  %\<currency>

% this should be the last package used
\usepackage{pdfsetup}

% urls in roman style, theory text in math-similar italics
\urlstyle{rm}
\isabellestyle{it}

% for uniform font size
%\renewcommand{\isastyle}{\isastyleminor}


\begin{document}

\title{Priority Search Trees}
\author{Peter Lammich \and Tobias Nipkow}
\maketitle

\begin{abstract}
We present a new, purely functional, simple and efficient data structure
combining a search tree and a priority queue, which we call a \emph{priority search tree}.
The salient feature of priority search trees is that they offer
a decrease-key operation, something that is missing from other simple, purely functional priority queue
implementations. Priority search trees can be implemented on top of any search tree.
This entry does the implementation for red-black trees.

This entry formalizes the first part of our ITP-2019 proof
pearl \emph{Purely Functional, Simple and Efficient Priority Search Trees and Applications to Prim and Dijkstra}
~\cite{LaNi19}.
\end{abstract}

\clearpage

\tableofcontents

% sane default for proof documents
\parindent 0pt\parskip 0.5ex

% generated text of all theories
\input{session}

\section{Related Work}
Our priority map ADT is close to Hinze's \cite{DBLP:conf/icfp/Hinze01} \emph{priority search queue}
interface, except that he also supports a few further operations that we could easily add but do not
need for our applications. However, it is not clear if his implementation technique is the same as
our priority search tree because his description employs a plethora of concepts, e.g.\
\emph{priority search pennants}, \emph{tournament trees}, \emph{semi-heaps}, and multiple
\emph{views} of data types that obscure a direct comparison. We claim that at the very least our
presentation is new because it is much simpler; we encourage the reader to compare the two.

As already observed by Hinze, McCreight's \cite{McCreight85} priority search trees support range
queries more efficiently than our trees. However, we can support the same range queries as Hinze
efficiently, but that is outside the scope of this entry.

% optional bibliography
\bibliographystyle{abbrv}
\bibliography{root}

\end{document}

%%% Local Variables:
%%% mode: latex
%%% TeX-master: t
%%% End:
