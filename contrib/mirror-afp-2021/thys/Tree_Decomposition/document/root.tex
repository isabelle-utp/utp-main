\documentclass[11pt,a4paper]{scrartcl}
\usepackage{isabelle,isabellesym}
\usepackage{amsmath,amsfonts}
\usepackage[utf8]{inputenc}
\usepackage[T1]{fontenc}

\typearea{11}
\renewcommand{\bf}{\normalfont\bfseries}
\renewcommand{\rm}{\normalfont\rmfamily}
\renewcommand{\it}{\normalfont\itshape}

\usepackage{pdfsetup}

% urls in roman style, theory text in math-similar italics
\urlstyle{rm}
\isabellestyle{it}

% for uniform font size
%\renewcommand{\isastyle}{\isastyleminor}

\begin{document}

\title{Tree Decompositions}
\author{Christoph Dittmann\\christoph.dittmann@tu-berlin.de}
\date{\today}
\maketitle

\begin{abstract}
  We formalize tree decompositions and tree width in Isabelle/HOL,
  proving that trees have treewidth~1.  We also show that every edge
  of a tree decomposition is a separation of the underlying graph.  As
  an application of this theorem we prove that complete graphs of size
  $n$ have treewidth $n-1$.
\end{abstract}

\tableofcontents
\newpage

\section{Introduction}

We follow \cite{diestel2006} in terms of the definition of tree
decompositions and treewidth.  We write a fairly minimal formalization
of graphs and trees and then go straight to tree decompositions.

Let $G = (V,E)$ be a graph and $(\mathcal{T},\beta)$ be a tree
decomposition, where $\mathcal{T}$ is a tree and $\beta:
V(\mathcal{T}) \to 2^V$ maps bags to sets of vertices.  Our main
theorem is that if $(s,t) \in V(\mathcal{T})$ is an edge of the tree
decomposition, then $\beta(s) \cap \beta(t)$ is a separator of $G$,
separating
\[ \bigcup_{\text{$u \in V(T)$ is in the left subtree of $\mathcal{T}
      \setminus (s,t)$}} \beta(u) \] and
\[ \bigcup_{\text{$u \in V(T)$ is in the right subtree of $\mathcal{T}
      \setminus (s,t)$}} \beta(u). \]

As an application of this theorem we show that if $K_n$ is the
complete graph on $n$ vertices, then the treewidth of $K_n$ is $n-1$.

Independent of this theorem, relying only on the basic definitions of
tree decompositions, we also prove that trees have treewidth 1 if they
have at least one edge (and treewidth 0 otherwise, which is trivial
and holds for all graphs).

\subsection{Avoid List Indices}

While this will be obvious for more experienced Isabelle/HOL users,
what we learned in this work is that working with lists becomes
significantly easier if we avoid indices.  It turns out that indices
often trip up Isabelle's automatic proof methods.  Rewriting a proof
with list indices to a proof without often reduced the length of the
proof by 50\% or more.

For example, instead of saying ``let $n \in \mathbb{N}$ be maximal
such that the first $n$ elements of the list all satisfy property
$P$'', it is better to say ``let $ps$ be a maximal prefix such that
all elements of $ps$ satisfy $P$''.

\subsection{Future Work}

We have several ideas for future work.  Let us enumerate them in order
of ascending difficulty (subjectively, of course).
\begin{enumerate}
\item The easiest would be a formalization of the fact that treewidth
  is closed under minors and disjoint union, and that adding a single
  edge increases the treewidth by at most one.  There are probably
  many more theorems similar to these.
\item A more interesting project would be a formalization of the cops
  and robber game for treewidth, where the number of cops is
  equivalent to the treewidth plus one.  See \cite{fomin2008} for a
  survey on these games.
\item Another interesting project would be a formal proof that the
  treewidth of a square grid is large.  It seems reasonable to expect
  that this could profit from a formalization of cops and robber
  games, but it is no prerequisite.
\item An ambitious long-term project would be a full formalization of
  the grid theorem by Robertson and Seymour
  \cite{robertson_seymour_graphs/V}.  They showed that there exists a
  function $f: \mathbb{N} \to \mathbb{N}$ such that for every $k \in
  \mathbb{N}$ it holds that if a graph has treewidth at least $f(k)$,
  then it contains a $k \times k$ grid as a minor.
\end{enumerate}

Another more technical point would be to evaluate whether it would be
good to use the ``Graph Theory'' library \cite{Graph_Theory-AFP} from
the Archive of Formal Proofs instead of reimplementing graphs here.
At first glance it seems that the graph theory library would provide a
lot of helpful lemmas.  On the other hand, it would be a non-trivial
dependency with its own idiosyncrasies, which could complicate the
development of tree decomposition proofs.  The author feels that
overall it is probably a good idea to base this work on the graph
theory library, but it needs further consideration.

% sane default for proof documents
\parindent 0pt\parskip 0.5ex

% generated text of all theories
\input{session}

\clearpage
\phantomsection
\addcontentsline{toc}{section}{Bibliography}
\bibliographystyle{plain}
\bibliography{root}

\end{document}

%%% Local Variables:
%%% mode: latex
%%% TeX-master: t
%%% End:
