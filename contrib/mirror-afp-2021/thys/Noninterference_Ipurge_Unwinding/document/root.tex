\documentclass[11pt,a4paper]{article}
\usepackage{isabelle,isabellesym}
\renewcommand{\isastyletxt}{\isastyletext}

% this should be the last package used
\usepackage{pdfsetup}

% urls in roman style, theory text in math-similar italics
\urlstyle{rm}
\isabellestyle{it}


\begin{document}

\title{The Ipurge Unwinding Theorem\\for CSP Noninterference Security}
\author{Pasquale Noce\\Security Certification Specialist at Arjo Systems - Gep S.p.A.\\pasquale dot noce dot lavoro at gmail dot com\\pasquale dot noce at arjowiggins-it dot com}
\maketitle

\begin{abstract}
The definition of noninterference security for Communicating Sequential
Processes requires to consider any possible future, i.e. any indefinitely long
sequence of subsequent events and any indefinitely large set of refused events
associated to that sequence, for each process trace. In order to render the
verification of the security of a process more straightforward, there is a need
of some sufficient condition for security such that just individual accepted and
refused events, rather than unbounded sequences and sets of events, have to be
considered.

Of course, if such a sufficient condition were necessary as well, it would be
even more valuable, since it would permit to prove not only that a process is
secure by verifying that the condition holds, but also that a process is not
secure by verifying that the condition fails to hold.

This paper provides a necessary and sufficient condition for CSP noninterference
security, which indeed requires to just consider individual accepted and refused
events and applies to the general case of a possibly intransitive policy. This
condition follows Rushby's output consistency for deterministic state machines
with outputs, and has to be satisfied by a specific function mapping security
domains into equivalence relations over process traces. The definition of this
function makes use of an intransitive purge function following Rushby's one;
hence the name given to the condition, Ipurge Unwinding Theorem.

Furthermore, in accordance with Hoare's formal definition of deterministic
processes, it is shown that a process is deterministic just in case it is a
trace set process, i.e. it may be identified by means of a trace set alone,
matching the set of its traces, in place of a failures-divergences pair. Then,
variants of the Ipurge Unwinding Theorem are proven for deterministic processes
and trace set processes.
\end{abstract}

\tableofcontents

% sane default for proof documents
\parindent 0pt\parskip 0.5ex

% generated text of all theories
\input{session}

% bibliography
\bibliographystyle{abbrv}
\bibliography{root}

\end{document}
