\documentclass[11pt,a4paper]{report}
\usepackage{isabelle,isabellesym}
\usepackage[only,bigsqcap]{stmaryrd}  % for \<Sqinter>

% this should be the last package used
\usepackage{pdfsetup}

% urls in roman style, theory text in math-similar italics
\urlstyle{rm}
\isabellestyle{it}


\begin{document}

\title{The Hereditarily Finite Sets}
\author{Lawrence C. Paulson}
\maketitle

\begin{abstract}
The theory of hereditarily finite sets is formalised, following 
the development of {\'S}wierczkowski \cite{swierczkowski-finite}.
An HF set is a finite collection of other HF sets; they enjoy an induction principle
and satisfy all the axioms of ZF set theory apart from the axiom of infinity, which is negated.
All constructions that are possible in ZF set theory (Cartesian products, disjoint sums, natural numbers,
functions) without using infinite sets are possible here.
The definition of addition for the HF sets follows Kirby \cite{kirby-addition}.

This development forms the foundation for the Isabelle proof of G\"odel's incompleteness theorems,
which has been formalised separately. 
\end{abstract}

\tableofcontents

% include generated text of all theories
\input{session}

\bibliographystyle{abbrv}
\bibliography{root}

\end{document}
