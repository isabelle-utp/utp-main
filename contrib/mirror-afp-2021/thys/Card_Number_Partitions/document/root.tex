\documentclass[11pt,a4paper]{article}
\usepackage{isabelle,isabellesym}

% this should be the last package used
\usepackage{pdfsetup}

% urls in roman style, theory text in math-similar italics
\urlstyle{rm}
\isabellestyle{it}


\begin{document}

\title{Cardinality of Number Partitions}
\author{Lukas Bulwahn}
\maketitle

\begin{abstract}
This entry provides a basic library for number partitions, defines the
two-argument partition function through its recurrence relation and relates
this partition function to the cardinality of number partitions. The main
proof shows that the recursively-defined partition function with arguments
$n$ and $k$ equals the cardinality of number partitions of $n$ with exactly
$k$ parts. The combinatorial proof follows the proof sketch of Theorem~2.4.1
in Mazur's textbook ``Combinatorics: A Guided Tour''~\cite{mazur-2010}. This
entry can serve as starting point for various more intrinsic properties about
number partitions, the partition function and related recurrence relations.

\end{abstract}

\tableofcontents

% sane default for proof documents
\parindent 0pt\parskip 0.5ex

% generated text of all theories
\input{session}

\nocite{*}

\bibliographystyle{abbrv}
\bibliography{root}

\end{document}

%%% Local Variables:
%%% mode: latex
%%% TeX-master: t
%%% End:
