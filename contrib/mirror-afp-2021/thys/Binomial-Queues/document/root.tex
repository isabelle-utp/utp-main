
\documentclass[fleqn]{llncs}

\usepackage[english]{babel}
\usepackage{isabelle}
\usepackage{isabellesym}
\usepackage{haftmann}
\usepackage{stmaryrd}
\usepackage{tabularx}
\usepackage{amssymb,amsmath}


%% format

\pagestyle{plain}
\isabellestyle{it}
\renewcommand{\isastyle}{\isastyleminor} % for uniform font size


%% style

% pipe bar with same width as space
\renewcommand{\isacharbar}{\isamath{\mid}\hspace{0.079em}}

\renewcommand{\isamarkupsubsubsection}[1]{\subsubsection{#1} ~ \\ \par }


%% hyphenation

\hyphenation{Isabelle}


%% document infos

\title{Verification of Functional Binomial Queues}

\author{Ren\'{e} Neumann}
\institute{Technische Universit\"at M\"unchen, Institut f\"ur Informatik \\
  \url{http://www.in.tum.de/~neumannr/}}


%% document

\begin{document}

\maketitle

\begin{abstract}
Priority queues are an important data structure and efficient implementations of them are crucial.  We implement a functional variant of binomial queues in Isabelle/HOL and show its functional correctness.  A verification against an abstract reference specification of priority queues has also been attempted, but could not be achieved to the full extent.
\end{abstract}

\vspace*{1ex}

\input{session}

\vspace*{-3ex}

\bibliographystyle{spmpsci}
\bibliography{root}

\end{document}
