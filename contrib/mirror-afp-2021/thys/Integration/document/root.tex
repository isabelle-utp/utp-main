\documentclass[11pt,a4paper]{report}
\usepackage{latexsym}
\usepackage{amsmath}
\usepackage{amssymb}
\usepackage[english]{babel}
\usepackage[utf8]{inputenc}
\usepackage[only,bigsqcap]{stmaryrd}
\usepackage{wasysym}
\usepackage{eufrak}
\usepackage{textcomp}
%\usepackage{apalike}
%\usepackage{times}

\usepackage{isabelle,isabellesym} 

% this should be the last package used
\usepackage{pdfsetup}

% proper setup for best-style documents
\urlstyle{rm}   
\isabellestyle{it}

\hyphenation{Isabelle}

\date{15th May 2003}
\parindent 0pt\parskip 0.5ex

\begin{document}


\title{Formalizing Integration Theory, with an Application to
  Probabilistic Algorithms}


\author{Stefan Richter\\
  LuFG Theoretische Informatik\\
  RWTH Aachen\\
  Ahornstraße 55\\
  52056 Aachen\\
  FRG\\
  \url{richter@informatik.rwth-aachen.de}}

\date{\today}
      
\maketitle

\pagestyle{headings}
\tableofcontents

% include generated text of all theories
% \input{session}
\newpage
\pagestyle{headings}
\section{Introduction}

Promela~\cite{Promela} is a modeling language, mainly used in the model checker SPIN~\cite{Holzmann03}. It offers a C-like syntax and allows to define processes to be run concurrently. Those processes can communicate via shared global variables or by message-passing via channels. Inside a process, constructs exist for non-deterministic choice, starting other processes and enforcing atomicity. It furthermore allows different means for specifying properties:  LTL formulae, assertions in the code, never claims (\ie an automata that explicitly specifies unwanted behavior) and others.

Some constructs found in Promela models, like \texttt{\#include} and \texttt{\#define}, are not part of the language Promela itself, but belong to the language of the C preprocessor. SPIN does not process those, but calls the C compiler internally to process them. We do not deal with them here, but also expect the sources to be preprocessed.

%Though there are approaches for giving a formal semantics of Promela~\cite{Weise:1997:promela,Gallardo:2004:promela,Sharma:2013:promela}, none of them shows that its definition matches reality. Moreover, some refer to outdated versions of the language.

Observing the output of SPIN and examining the generated graphs often is the only way of determining the semantics of a certain construct. This is complicated further by SPIN unconditionally applying optimizations. For the current formalization we chose to copy the semantics of SPIN, including the aforementioned optimizations. For some constructs, we had to restrict the semantics, \ie some models are accepted by SPIN, but not by this formalization. Those deviations are:
\begin{itemize}
    \item \texttt{run} is a statement instead of an expression. SPIN here has a complicated set of restrictions unto where \texttt{run} can occur inside an expression. The sole use of it is to be able to get the ID of a spawned process. We omitted this feature to guarantee expressions to be free of side-effects.
    \item Variable declarations which got jumped over are seen as not existing. In SPIN, such constructs show surprising behavior:\\\texttt{int~i;~goto~L;~i~=~5;~L:~printf("\%d",~i)} yields $0$, while \\\texttt{goto L; int i = 5; L: printf("\%d", i)} yields $5$.\\
        The latter is forbidden in our formalization (it will get rejected with ``unknown variable~i''), while the first behaves as in SPIN.
    \item Violating an \texttt{assert} does not abort, but instead sets the variable \texttt{\_\_assert\_\_} to true. This needs to be checked explicitly in the LTL formula. We plan on adding this check in an automatic manner.
    \item Types are bounded. Except for well-defined types like booleans, overflow is not allowed and will result in an error. The same holds for assigning a value that is outside the bounds. SPIN does not specify any explicit semantics here, but solely refers to the underlying C-compiler and its semantics. This might result in two models behaving differently on different systems when run with SPIN, while this formalization, due to the explicit bounds in the semantics, is not affected.
\end{itemize}

Additionally, some constructs are currently not supported, and the compilation will abort if they are encountered: \texttt{d\_step}\footnote{This can be safely replaced by \texttt{atomic}, though larger models will be produced then.}, \texttt{typedef}, remote references, bit-operations, \texttt{unsigned}, 
and property specifications except \texttt{ltl} and \texttt{assert}. Other constructs are accepted but ignored, because they do not change the behavior of a model: advanced variable scoping, \texttt{xr}, \texttt{xs}, \texttt{print*}, priorities, and visibility of variables.

Nonetheless, for models not using those unsupported constructs, we generate the very same number of states as SPIN does. An exception applies for large \texttt{goto} chains and when simultaneous termination of multiple processes is involved, as SPIN's semantics is too vague here.

\input{Sigma_Algebra}  
\input{MonConv}
\input{Measure}
\newpage
\input{RealRandVar}
\chapter{Integration}
\label{cha:integration}

The chapter at hand assumes a central position in the present paper. The Lebesgue
integral is defined and its characteristics are shown in
\ref{sec:stepwise-approach}. To illustrate the problems arising in
doing so, we first look at implementation alternatives that did not
work out. 

\input{Failure}
\input{Integral}
\chapter{Epilogue}
\label{cha:epilogue}

To come to a conclusion, a few words shall
subsume the work done and point out opportunities for future research at the same time.

What has been achieved? After opening with some
introductory notes, we began translating the language of measure
theory into machine checkable text. For the material in
section \ref{sec:preliminaries}, this had been done before. Besides laying the
foundation for the development, the style of presentation should make it
noteworthy.  

It is a particularity of the present work that its theories are
written in the Isar language, a declarative proof language that aims
to be ``intelligible''. This is not a novelty, nor is it the author's
merit. Still, giving full formal proofs in a text intended to be read
by people is in a way experimental. Clearly, it is bound to put some
strain on the reader. Nevertheless, I hope that we have made a little
step towards formalizing mathematical knowledge in a way that is
equally suitable for computation and understanding. One aim of the
research done has been to demonstrate the viability of this approach.
Unquestionably, there is plenty room for improvement regarding the
quality of presentation. The language itself has, in my opinion,
proven to be fit for a wide range of applications, including the
classical mathematics we used it for.

Returning to a more content-centered viewpoint, we discussed the
measurability of real-valued functions in section \ref{sec:realrandvar}. As
explained there, earlier scholarship has resulted in related theories
for the MIZAR environment though the development seems to have
stopped. Anyway, the mathematics covered should be new to HOL-based
systems. 

More functions could obviously be demonstrated to be random
variables. We shortly commented on an alternative approach in the
section just mentioned. It is applicable to continuous functions,
proving these measurable all at once. Efforts on topological spaces
would be required, but they constitute an interesting field
themselves, so it is probably worth the while.

In the third chapter, integration in the Lebesgue style
has been looked at in depth. To my knowledge, no similar theory had
been developed in a theorem prover up to this point.  We managed to
systematically establish the integral of increasingly complex
functions. Simple or nonnegative functions ought to be treated in
sufficient detail by now. Of course, the repository of potential
supplementary facts is vast. Convergence theorems, as well as the
interrelationship with differentiation or concurrent integral concepts,
are but a few examples. They leave ample space for subsequent work.

A shortcoming of the present development lies in the lack of
user assistance. Greater care could be taken to ensure automatic
application of appropriate simplification rules --- or to design such
rules in the first place. Likewise, the principal requirement of
integrability might hinder easy usage of the integral. Fixing a
default value for undefined integrals could possibly make some case
distinctions obsolete. Facets like these have not been addressed in
their due extent.

 
\begin{flushleft}
\bibliographystyle{plain}
\bibliography{root}
\end{flushleft}

\end{document}
