\documentclass[11pt,a4paper]{article}
\usepackage{isabelle,isabellesym}
\renewcommand{\isastyletxt}{\isastyletext}

% this should be the last package used
\usepackage{pdfsetup}

% urls in roman style, theory text in math-similar italics
\urlstyle{rm}
\isabellestyle{it}


\begin{document}

\title{Reasoning about Lists via List Interleaving}
\author{Pasquale Noce\\Security Certification Specialist at Arjo Systems - Gep S.p.A.\\pasquale dot noce dot lavoro at gmail dot com\\pasquale dot noce at arjowiggins-it dot com}
\maketitle

\begin{abstract}
Among the various mathematical tools introduced in his outstanding work on
Communicating Sequential Processes, Hoare has defined "interleaves" as the
predicate satisfied by any three lists such that the first list may be split
into sublists alternately extracted from the other two ones, whatever is the
criterion for extracting an item from either one list or the other in each step.

This paper enriches Hoare's definition by identifying such criterion with the
truth value of a predicate taking as inputs the head and the tail of the first
list. This enhanced "interleaves" predicate turns out to permit the proof of
equalities between lists without the need of an induction. Some rules that allow
to infer "interleaves" statements without induction, particularly applying to
the addition or removal of a prefix to the input lists, are also proven.
Finally, a stronger version of the predicate, named "Interleaves", is shown to
fulfil further rules applying to the addition or removal of a suffix to the
input lists.
\end{abstract}

\tableofcontents

% sane default for proof documents
\parindent 0pt\parskip 0.5ex

% generated text of all theories
\input{session}

% bibliography
\bibliographystyle{abbrv}
\bibliography{root}

\end{document}
