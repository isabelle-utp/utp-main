\documentclass[11pt,a4paper]{article}
\usepackage{isabelle,isabellesym}

% this should be the last package used
\usepackage{pdfsetup}

% urls in roman style, theory text in math-similar italics
\urlstyle{rm}
\isabellestyle{it}


\begin{document}

\title{SAT Solver verification}
\author{By Filip Mari\' c}
\maketitle
\abstract{
This document contains formall correctness proofs of modern SAT solvers. 
Two different approaches are used --- state-transition systems and 
shallow embedding into HOL. 

Formalization based on state-transition systems follows 
\cite{KrsticGoel,NieuwenhuisOliverasTinelli}.
Several different SAT solver descriptions are given and their partial 
correctness and termination is proved. These include:
\begin{enumerate}
\item a solver based on classical DPLL procedure (based on backtrack-search with unit propagation), 
\item a very general solver with backjumping and learning (similiar to the description given in 
      \cite{NieuwenhuisOliverasTinelli}), and 
\item a solver with a specific conflict analysis algorithm (similiar to the description given 
      in \cite{KrsticGoel}). 
\end{enumerate}

Formalization based on shallow embedding into HOL defines a SAT solver 
as a set or recursive HOL functions. Solver supports most state-of-the 
art techniques including the two-watch literal propagation scheme.

Within the SAT solver correctness proofs, a large number of lemmas
about propositional logic and CNF formulae are proved. This theory is
self-contained and could be used for further exploring of properties of
CNF based SAT algorithms.  
}

\tableofcontents

% sane default for proof documents
\parindent 0pt\parskip 0.5ex

% generated text of all theories
\input{session}

% optional bibliography
\bibliographystyle{abbrv}
\bibliography{root}

\end{document}

%%% Local Variables:
%%% mode: latex
%%% TeX-master: t
%%% End:
