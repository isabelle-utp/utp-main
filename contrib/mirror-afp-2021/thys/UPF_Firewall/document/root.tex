\documentclass[11pt,DIV10,a4paper,twoside=semi,openright,titlepage]{scrreprt}
\usepackage{fixltx2e}
%%%%%%%%%%%%%%%%%%%%%%%%%%%%%%%%%%%%%%%%%%%%%%%%%%%%%%%%%%%%%%%%%%%%%%%
%%% Overrides the (rightfully issued) warning by Koma Script that \rm
%%% etc. should not be used (they are deprecated since more than a
%%% decade)
  \DeclareOldFontCommand{\rm}{\normalfont\rmfamily}{\mathrm}
  \DeclareOldFontCommand{\sf}{\normalfont\sffamily}{\mathsf}
  \DeclareOldFontCommand{\tt}{\normalfont\ttfamily}{\mathtt}
  \DeclareOldFontCommand{\bf}{\normalfont\bfseries}{\mathbf}
  \DeclareOldFontCommand{\it}{\normalfont\itshape}{\mathit}
%%%%%%%%%%%%%%%%%%%%%%%%%%%%%%%%%%%%%%%%%%%%%%%%%%%%%%%%%%%%%%%%%%%%%%%

\usepackage{isabelle,isabellesym}
\usepackage{stmaryrd}
\usepackage{paralist}
\usepackage{xspace}
\usepackage{amsmath}
\usepackage[english]{babel}
\newcommand{\testgen}{HOL-TestGen\xspace}
\newcommand{\testgenFW}{HOL-TestGen/FW\xspace}
\usepackage[numbers, sort&compress, sectionbib]{natbib}
\usepackage{graphicx}
\usepackage{color}
\sloppy

\usepackage{amssymb}
\newcommand{\isadefinition} {{\operatorname{definition}}}
\newcommand{\types} {{\operatorname{type\_synonym}}}
\newcommand{\datatype} {{\operatorname{datatype}}}
\newcommand{\ap}{\,}


\newcommand{\dom}{\mathrm{dom}}
\newcommand{\ran}{\mathrm{ran}}
\newcommand{\ofType}{\!::\!}
\newcommand{\HolBin}[0]{\ensuremath{\mathrm{bin}}}
\newcommand{\HolNum}[0]{\ensuremath{\mathrm{num}}}
\newcommand{\HolBoolean}[0]{\ensuremath{\mathrm{bool}}}
\newcommand{\HolString}[0]{\ensuremath{\mathrm{string}}}
\newcommand{\HolInteger}[0]{\ensuremath{\mathrm{int}}}
\newcommand{\HolNat}[0]{\ensuremath{\mathrm{nat}}}
\newcommand{\HolReal}[0]{\ensuremath{\mathrm{real}}}
\newcommand{\HolSet}[1]{#1\ap\ensuremath{\mathrm{set}}}
\newcommand{\HolList}[1]{#1\ap\ensuremath{\mathrm{list}}}
%\newcommand{\HolOrderedSet}[1]{#1~\ensuremath{\mathrm{orderedset}}}
\newcommand{\HolMultiset}[1]{#1\ap\ensuremath{\mathrm{multiset}}}
\newcommand{\classType}[2]{#1\ap\ensuremath{\mathrm{#2}}}
\newcommand{\bottom}{\bot}
\DeclareMathOperator{\HolSome}{Some}
\DeclareMathOperator{\HolNone}{None}
\DeclareMathOperator{\Poverride}{\oplus}
\DeclareMathOperator{\prodTwo}{\otimes_2}

\newcommand{\HolMkSet}[1]{\operatorname{set} #1}
\newcommand{\spot}{.\;}
\newcommand{\where} {{\operatorname{where}}}
\DeclareMathOperator{\HolIf}{if}
\DeclareMathOperator{\HolLet}{let}
\DeclareMathOperator{\HolIn}{in}
\DeclareMathOperator{\HolThen}{then}
\DeclareMathOperator{\HolElse}{else}


\newcommand{\isasymmodels}{\isamath{\models}}
\newcommand{\HOL}{HOL}

\newcommand{\ie}{i.\,e.}
\newcommand{\eg}{e.\,g.}

\usepackage{pdfsetup}

\urlstyle{rm}
\isabellestyle{it}
\renewcommand{\isastyle}{\isastyleminor}

\pagestyle{empty} 
\begin{document}
\renewcommand{\subsubsectionautorefname}{Section}
\renewcommand{\subsectionautorefname}{Section}
\renewcommand{\sectionautorefname}{Section}
\renewcommand{\chapterautorefname}{Chapter}
\newcommand{\subtableautorefname}{\tableautorefname}
\newcommand{\subfigureautorefname}{\figureautorefname}

\title{Formal Network Models and Their Application to Firewall Policies\\ (UPF-Firewall)}
\author{Achim D. Brucker\footnotemark[1] \quad
        Lukas Br\"ugger\footnotemark[2]  \quad
        Burkhart Wolff\footnotemark[3]\\[1.5em]
  \normalsize
  \normalsize\footnotemark[1]~Department of Computer Science, The University of Sheffield, Sheffield, UK
  \texorpdfstring{\\}{}
  \normalsize\href{mailto:"Achim D. Brucker"
    <a.brucker@sheffield.ac.uk>}{a.brucker@sheffield.ac.uk}\\[1em]
  %
  \normalsize\footnotemark[2]Information Security, ETH Zurich, 8092 Zurich, Switzerland
  \texorpdfstring{\\}{}
  \normalsize\href{mailto:"Lukas Bruegger"
    <lukas.a.bruegger@gmail.com>}{Lukas.A.Bruegger@gmail.com}\\[1em]
  %
  \normalsize\footnotemark[3]~Univ. Paris-Sud, Laboratoire LRI,
  UMR8623, 91405 Orsay, France
  France\texorpdfstring{\\}{}
  \normalsize\href{mailto:"Burkhart Wolff" <burkhart.wolff@lri.fr>}{burkhart.wolff@lri.fr}
}

\pagestyle{empty}
\publishers{%
  \normalfont\normalsize%
    \centerline{\textsf{\textbf{\large Abstract}}}
    \vspace{1ex}%
    \parbox{0.8\linewidth}{%
      We present a formal model of network protocols and their
      application to modeling firewall policies. The formalization is
      based on the \emph{Unified Policy Framework} (UPF). The
      formalization was originally developed with for generating test
      cases for testing the security configuration actual firewall and
      router (middle-boxes) using HOL-TestGen. Our work focuses on
      modeling application level protocols on top of tcp/ip. 
    }
}

\maketitle
\cleardoublepage
\pagestyle{plain}
\tableofcontents
\cleardoublepage

\chapter{Introduction}
  \section{Introduction}

This document is based on
\cite{ArkoudasETAL04VerifyingFileSystemImplementationICFEM}, which
explores the challenges of verifying the core operations of a
Unix-like file system \cite{thompson78unix,mckusick84fast}.  The paper
\cite{ArkoudasETAL04VerifyingFileSystemImplementationICFEM} formalizes
the specification of the file system as a map from file names to
sequences of bytes, then formalizes an implementation that uses such
standard file system data structures as i-nodes and fixed-sized disk
blocks.  The correctness of the
implementation is verified by proving the existence of a simulation relation
\cite{RoeverEngelhardt98DataRefinement} between the specification and
the implementation.  The original effort of
\cite{ArkoudasETAL04VerifyingFileSystemImplementationICFEM} started in
Isabelle.  The process of developing the proof in Isabelle helped to 
remove the initial bugs in the concrete and
abstract models (though the proof has not been completed so far).  

Here we present a completed proof for a simplified problem:
data refinement of a single file.  We present operations on
both abstract and concrete files, define a function mapping
concrete files to abstract files, and prove that this
function is a simulation relation.

We use two libraries of arrays: arrays without bounds
checks, which can be thought of as an array with an
unbounded number of elements, and resizable arrays, which
have a notion of the current size, but expand in response to
array writes that are outside the current bounds.

%%%%%%%%%%%%%%%%%%%%%%%%%%%%%%
% <session>
  % \input{session}
  \input{UPF-Firewall}
    \input{NetworkModels}
      \input{NetworkCore}
      \input{DatatypeAddress}
      \input{DatatypePort}
      \input{IntegerAddress}
      \input{IntegerPort}
      \input{IntegerPort_TCPUDP}
      \input{IPv4}
      \input{IPv4_TCPUDP.tex}
    \input{PacketFilter.tex}
      \input{PolicyCore}
      \input{PolicyCombinators}
      \input{PortCombinators}
      \input{ProtocolPortCombinators}
      \input{Ports}
      \input{NAT}
    \input{FWNormalisation.tex}
      \input{FWNormalisationCore.tex}
      \input{NormalisationGenericProofs.tex}
      \input{NormalisationIntegerPortProof.tex}
      \input{NormalisationIPPProofs.tex}
    \input{StatefulFW}
      \input{StatefulCore}
      \input{FTP}
      \input{FTP_WithPolicy}
      \input{VOIP}
      \input{FTPVOIP}
  %%%%%%%%%%%%%%%%%%%%%%%%%%%%%
  \input{Examples.tex}
    \input{DMZ.tex}
      \input{DMZDatatype.tex}
      \input{DMZInteger.tex}
    \input{PersonalFirewall.tex}
      \input{PersonalFirewallInt.tex}
      \input{PersonalFirewallIpv4.tex}
      \input{PersonalFirewallDatatype.tex}
    \input{Transformation.tex}
      \input{Transformation01.tex}
      \input{Transformation02.tex}
    \input{NAT-FW.tex}
    \input{Voice_over_IP.tex}
% </session>
%%%%%%%%%%%%%%%%%%%%%%%%%%%%%%

\bibliographystyle{abbrvnat}
\bibliography{root}
\end{document}

%%% Local Variables:
%%% mode: latex
%%% TeX-master: t
%%% End:
