\documentclass[11pt,a4paper]{article}
\usepackage{isabelle,isabellesym}

% further packages required for unusual symbols (see also
% isabellesym.sty), use only when needed

\usepackage{amssymb}
  %for \<leadsto>, \<box>, \<diamond>, \<sqsupset>, \<mho>, \<Join>,
  %\<lhd>, \<lesssim>, \<greatersim>, \<lessapprox>, \<greaterapprox>,
  %\<triangleq>, \<yen>, \<lozenge>

%\usepackage{eurosym}
  %for \<euro>

%\usepackage[only,bigsqcap]{stmaryrd}
  %for \<Sqinter>

\usepackage{eufrak}
  %for \<AA> ... \<ZZ>, \<aa> ... \<zz> (also included in amssymb)

%\usepackage{textcomp}
  %for \<onequarter>, \<onehalf>, \<threequarters>, \<degree>, \<cent>,
  %\<currency>

% this should be the last package used
\usepackage{pdfsetup}

% urls in roman style, theory text in math-similar italics
\urlstyle{rm}
\isabellestyle{it}

% for uniform font size
%\renewcommand{\isastyle}{\isastyleminor}

\renewcommand{\refname}{Bibliography}

\begin{document}

\title{Chamber complexes, Coxeter systems, and buildings}
\author{Jeremy Sylvestre \\ University of Alberta, Augustana Campus \\ \href{mailto:jeremy.sylvestre@ualberta.ca}{\url{jeremy.sylvestre@ualberta.ca}}}
\maketitle

\begin{abstract}
We provide a basic formal framework for the theory of chamber complexes and Coxeter systems, and for buildings as thick chamber complexes endowed with a system of apartments. Along the way, we develop some of the general theory of abstract simplicial complexes and of groups (relying on the \textit{group{\_}add} class for the basics), including free groups and group presentations, and their universal properties. The main results verified are that the deletion condition is both necessary and sufficient for a group with a set of generators of order two to be a Coxeter system, and that the apartments in a (thick) building are all uniformly Coxeter.
\end{abstract}

\tableofcontents

% sane default for proof documents
\parindent 0pt\parskip 0.5ex

\vspace*{32pt}
\textit{Note:} A number of the proofs in this theory were modelled on or inspired by proofs in the books on buildings by Abramenko and Brown \cite{Abramenko+Brown:Buildings} and by Garrett \cite{Garrett:Buildings}. As well, some of the definitions, statments, and proofs appearing in the first two sections previously appeared in a submission to the \textit{Archive of Formal Proofs} by the author of the current submission \cite{Sylvestre-AFP15}.
\vspace*{32pt}

% generated text of all theories
\input{session}

\clearpage

% optional bibliography
\nocite{Johnson:GroupPres}
\bibliographystyle{abbrv}
\bibliography{root}

\end{document}

%%% Local Variables:
%%% mode: latex
%%% TeX-master: t
%%% End:
