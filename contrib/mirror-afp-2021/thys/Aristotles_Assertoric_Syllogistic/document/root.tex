\documentclass[11pt,a4paper]{article}
\usepackage{isabelle,isabellesym}

\usepackage{mathtools,url}
\usepackage{amssymb}

% this should be the last package used
\usepackage{pdfsetup}

% urls in roman style, theory text in math-similar italics
\urlstyle{rm}
\isabellestyle{it}

\bibliographystyle{plain}

\begin{document}

\title{Aristotle's Assertoric Syllogistic}
\author{Angeliki Koutsoukou-Argyraki}
\maketitle

\begin{abstract}
	We formalise with Isabelle/HOL some basic elements of Aristotle's assertoric syllogistic following
	the article from the Stanford Encyclopedia of Philosophy by Robin Smith:
	\url{https://plato.stanford.edu/entries/aristotle-logic/}.
	To this end, we use a set theoretic formulation (covering both individual and general predication).
	In particular, we formalise the deductions in the Figures and after that we present Aristotle's
	metatheoretical observation that all deductions in the Figures can in fact be reduced to either
	Barbara or Celarent. As the formal proofs prove to be straightforward, the interest of this entry 
	lies in illustrating the functionality of Isabelle and high efficiency of Sledgehammer for simple 
	exercises in philosophy.
\end{abstract}

\tableofcontents

% include generated text of all theories
\input{session}

%\bibliographystyle{abbrv}
%\bibliography{root}

\end{document}
