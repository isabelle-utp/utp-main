\documentclass[11pt,a4paper]{article}
\usepackage{isabelle,isabellesym}
\usepackage{amsmath}
\usepackage{amssymb}

% this should be the last package used
\usepackage{pdfsetup}

% urls in roman style, theory text in math-similar italics
\urlstyle{rm}
\isabellestyle{it}


\begin{document}

\title{Count the Number of Complex Roots}
\author{Wenda Li}
\maketitle

\begin{abstract}
  Based on evaluating Cauchy indices through remainder sequences \cite{eisermann2012fundamental} \cite[Chapter 11]{rahman2002analytic}, this entry provides an effective procedure to count the number of complex roots (with multiplicity) of a polynomial within a rectangle box or a half-plane. Potential applications of this entry include certified complex root isolation (of a polynomial) and testing the Routh-Hurwitz stability criterion (i.e., to check whether all the roots of some characteristic polynomial have negative real parts).
\end{abstract}

%\tableofcontents

% include generated text of all theories
\input{session}

\section{Acknowledgements}
The work was supported by the ERC Advanced Grant ALEXANDRIA (Project 742178), funded by the European Research Council
and led by Professor Lawrence Paulson at the University of Cambridge, UK.

\bibliographystyle{abbrv}
\bibliography{root}

\end{document}
