\documentclass[11pt,a4paper]{article}
\usepackage{isabelle,isabellesym}

\usepackage{amsmath}
\usepackage{amssymb}
\usepackage{amsthm}
\usepackage{xspace}
\usepackage[utf8]{inputenc}

% this should be the last package used
\usepackage{pdfsetup}

% urls in roman style, theory text in math-similar italics
\urlstyle{rm}
\isabellestyle{it}

\newtheorem{theorem}{Theorem}%[section]
\newtheorem{corollary}{Corollary}%[section]
\newcommand\isafor{\textsf{IsaFoR}}
\newcommand\ceta{\textsf{Ce\kern-.18emT\kern-.18emA}}
\newcommand\rats{\mathbb{Q}}
\newcommand\ints{\mathbb{Z}}
\newcommand\reals{\mathbb{R}}
\newcommand\complex{\mathbb{C}}

\begin{document}

\title{Stochastic Matrices and the Perron--Frobenius Theorem\footnote{Supported by FWF (Austrian Science Fund) project Y757.}}
\author{Ren\'e Thiemann}
\maketitle

\begin{abstract}
Stochastic matrices are a convenient way to model discrete-time and finite
state Markov chains. The Perron--Frobenius theorem tells us something about
the existence and uniqueness of non-negative eigenvectors of a stochastic matrix.

In this entry, we formalize stochastic matrices, link the formalization 
to the existing AFP-entry on Markov chains, and apply the Perron--Frobenius
theorem to prove that stationary distributions always exist, and they are
unique if the stochastic matrix is irreducible. 
\end{abstract}

\tableofcontents

\section{Introduction}

In their AFP entry Markov Models \cite{Markov_Models-AFP}, H\"olzl and Nipkow
provide a framework for specifying discrete- and continuous-time Markov chains.

In the following, we instantiate their framework by formalizing right-stochastic
matrices and stochastic vectors. These vectors encode probability mass functions
over a finite set of states, whereas stochastic matrices can be utilized to model 
discrete-time and finite space Markov chains. 

The formulation of Markov chains as matrices has the 
advantage that certain concepts
can easily be expressed via matrices. For instance, a stationary distribution is nothing
else than a non-negative real eigenvector of the transition matrix for eigenvalue 1.
As a consequence, we can derive certain properties on Markov chains using results on
matrices. To be more precise, we utilize the formalization of the Perron--Frobenius theorem
\cite{Perron_Frobenius-AFP} to prove that a stationary distribution always exists, and that it is unique if
the transition matrix is irreducible. 


\input{session}

\bibliographystyle{abbrv}
\bibliography{root}

\end{document}
