\documentclass[11pt,a4paper]{article}
\usepackage{isabelle,isabellesym}
\usepackage{amsfonts, amsmath, amssymb}

% this should be the last package used
\usepackage{pdfsetup}

% urls in roman style, theory text in math-similar italics
\urlstyle{rm}
\isabellestyle{it}

\begin{document}

\title{Gaussian Integers}
\author{Manuel Eberl}
\maketitle

\begin{abstract}
The Gaussian integers are the subring $\mathbb{Z}[i]$ of the complex numbers, i.\,e.\ the ring of all complex numbers with integral real and imaginary part. This article provides a definition of this ring as well as proofs of various basic properties, such as that they form a Euclidean ring and a full classification of their primes. An executable (albeit not very efficient) factorisation algorithm is also provided.

Lastly, this Gaussian integer formalisation is used in two short applications:
\begin{enumerate}
\item The characterisation of all positive integers that can be written as sums of two squares
\item Euclid's formula for primitive Pythagorean triples
\end{enumerate}
While elementary proofs for both of these are already available in the AFP, the theory of Gaussian integers provides more concise proofs and a more high-level view.
\end{abstract}

\newpage
\tableofcontents
\newpage
\parindent 0pt\parskip 0.5ex

\input{session}

\bibliographystyle{abbrv}
\bibliography{root}

\end{document}

%%% Local Variables:
%%% mode: latex
%%% TeX-master: t
%%% End:
