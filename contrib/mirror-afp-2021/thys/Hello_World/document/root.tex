\documentclass[11pt,a4paper]{article}
\usepackage{isabelle,isabellesym}

\usepackage{marvosym} % world symbol
\newcommand{\isactrlurl}[0]{\Mundus}

\usepackage{verbatim}


% this should be the last package used
\usepackage{pdfsetup}

% urls in roman style, theory text in math-similar italics
\urlstyle{rm}
\isabellestyle{it}


\begin{document}

\title{Hello World}
\author{Cornelius Diekmann, Lars Hupel}
\maketitle

\begin{abstract}
In this article, we present a formalization of the well-known ``Hello, World!'' code, including a formal framework for reasoning about IO.
Our model is inspired by the handling of IO in Haskell.
We start by formalizing the \isactrlurl{} and embrace the IO monad afterwards.
Then we present a sample \verb~main :: IO ()~, followed by its proof of correctness.
\end{abstract}

\tableofcontents

% sane default for proof documents
\parindent 0pt\parskip 0.5ex

% generated text of all theories
\input{session}

% optional bibliography
%\bibliographystyle{abbrv}
%\bibliography{root}

\end{document}

%%% Local Variables:
%%% mode: latex
%%% TeX-master: t
%%% End:
