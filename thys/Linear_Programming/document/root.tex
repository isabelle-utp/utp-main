\documentclass[11pt,a4paper]{article}
\usepackage{isabelle,isabellesym}

% further packages required for unusual symbols (see also
% isabellesym.sty), use only when needed

\usepackage{amssymb}
  %for \<leadsto>, \<box>, \<diamond>, \<sqsupset>, \<mho>, \<Join>,
  %\<lhd>, \<lesssim>, \<greatersim>, \<lessapprox>, \<greaterapprox>,
  %\<triangleq>, \<yen>, \<lozenge>

%\usepackage{eurosym}
  %for \<euro>

%\usepackage[only,bigsqcap]{stmaryrd}
  %for \<Sqinter>

%\usepackage{eufrak}
  %for \<AA> ... \<ZZ>, \<aa> ... \<zz> (also included in amssymb)

%\usepackage{textcomp}
  %for \<onequarter>, \<onehalf>, \<threequarters>, \<degree>, \<cent>,
  %\<currency>

% this should be the last package used
\usepackage{pdfsetup}

% urls in roman style, theory text in math-similar italics
\urlstyle{rm}
\isabellestyle{it}

% for uniform font size
%\renewcommand{\isastyle}{\isastyleminor}


\begin{document}

\title{Linear-Programming}
\author{Julian Parsert}

\maketitle

\begin{abstract}
We use the previous formalization of the general simplex algorithm to
formulate an algorithm for solving linear programs.
We encode the linear programs using only linear constraints.
Solving these constraints also solves the original linear program.
This algorithm is proven to be sound by applying the weak duality theorem which is also part of this formalization~\cite{schrijver1998theory}.
\end{abstract}

\tableofcontents

% sane default for proof documents
\parindent 0pt\parskip 0.5ex

\section{Related work}
Our work is based on a formalization of the
general simplex algorithm described
in~\cite{SimplexAFP,Spasic:FormIncrSimplex}. However, the general
simplex algorithm lacks the ability to optimize a function. Boulmé and
Maréchal~\cite{Sylvain:CoqTacForEqualityLinArith} describe a
formalization and implementation of Coq tactics for linear integer
programming and linear arithmetic over rationals. More closely related
is the formalization by Allamigeon et
al.~\cite{Allamigeon:FormCvxPolyhedraSimplex} which formalizes the
simplex method and related results. As part of Flyspeck project Obua
and Nipkow~\cite{Obua2009} created a verification mechanism for linear
programs using the HOL computing library and external solvers.

% generated text of all theories
\input{session}

% optional bibliography
\bibliographystyle{abbrv}
\bibliography{root}

\end{document}

%%% Local Variables:
%%% mode: latex
%%% TeX-master: t
%%% End:
