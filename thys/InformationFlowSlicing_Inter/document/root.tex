\documentclass[11pt,a4paper,notitlepage]{article}
\usepackage{isabelle,isabellesym}
\usepackage{latexsym}
\usepackage{amssymb}
\usepackage{textcomp}
\usepackage[english]{babel}
\usepackage[utf8]{inputenc}
\usepackage{wasysym}
\usepackage{graphicx}
\usepackage{url}

% this should be the last package used
\usepackage{pdfsetup}

% urls in roman style, theory text in math-similar italics
\urlstyle{rm}
\isabellestyle{it}


\begin{document}

\title{Slicing Guarantees Information Flow Noninterference}
\author{Daniel Wasserrab}
\maketitle

\begin{abstract}
  In this contribution, we show how correctness proofs for intra-
  \cite{Wasserrab:08} and interprocedural slicing \cite{Wasserrab:09}
  can be used to prove that slicing is able to guarantee information
  flow noninterference.  Moreover, we also illustrate how to lift the
  control flow graphs of the respective frameworks such that they
  fulfil the additional assumptions needed in the noninterference
  proofs. A detailed description of the intraprocedural proof and its
  interplay with the slicing framework can be found in
  \cite{WasserrabLS:09}.
\end{abstract}

\section{Introduction}
Information Flow Control (IFC) encompasses algorithms which determines
if a given program leaks secret information to public entities. The major 
group are so called IFC type systems, where well-typed means that the
respective program is secure. Several IFC type systems have been verified
in proof assistants, e.g.\ see \cite{BartheN:04,BarthePR:07,Kammueller:08,BeringerH:08,SneltingW:08}.

However, type systems have some drawbacks which can lead to false alarms.
To overcome this problem, an IFC approach basing on slicing has been developed
\cite{HammerS:09}, which can significantly reduce the amount of false alarms. 
This contribution presents the first machine-checked proof
that slicing is able to guarantee IFC noninterference. It bases on previously
published machine-checked correctness proofs for slicing 
\cite{Wasserrab:08,Wasserrab:09}. Details for the intraprocedural case can
be found in \cite{WasserrabLS:09}.



%\parindent 0pt\parskip 0.5ex

% generated text of all theories
\input{session}

% optional bibliography
\bibliographystyle{abbrv}
\bibliography{root}


\end{document}

%%% Local Variables:
%%% mode: latex
%%% TeX-master: t
%%% End:
