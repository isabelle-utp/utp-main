\documentclass[11pt,a4paper]{article}
\usepackage{isabelle,isabellesym}

% this should be the last package used
\usepackage{pdfsetup}

% urls in roman style, theory text in math-similar italics
\urlstyle{rm}
\isabellestyle{it}


\begin{document}

\title{Nominal 2}
\author{Christian Urban, Stefan Berghofer, and Cezary Kaliszyk}
\maketitle

\begin{abstract}
  Dealing with binders, renaming of bound variables, capture-avoiding
  substitution, etc., is very often a major problem in formal
  proofs, especially in proofs by structural and rule
  induction. Nominal Isabelle is designed to make such proofs easy to
  formalise: it provides an infrastructure for declaring nominal
  datatypes (that is alpha-equivalence classes) and for defining
  functions over them by structural recursion. It also provides
  induction principles that have Barendregt’s variable convention
  already built in.

  This entry can be used as a more advanced replacement for
  HOL/Nominal in the Isabelle distribution.
\end{abstract}

\tableofcontents

% include generated text of all theories
\input{session}

\bibliographystyle{abbrv}
\bibliography{root}

\end{document}
