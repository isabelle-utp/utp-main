\documentclass[11pt,a4paper]{article}
\usepackage{isabelle,isabellesym}

\usepackage{amssymb}
\usepackage{stmaryrd}

% this should be the last package used
\usepackage{pdfsetup}

% urls in roman style, theory text in math-similar italics
\urlstyle{rm}
\isabellestyle{it}

% for uniform font size
%\renewcommand{\isastyle}{\isastyleminor}


\begin{document}

\title{Zermelo Fraenkel Set Theory in Higher-Order Logic}
\author{Lawrence C. Paulson\\ Computer Laboratory\\ University of Cambridge}

\maketitle

\begin{abstract}
This entry is a new formalisation of ZFC set theory in Isabelle/HOL\@.
It is logically equivalent to Obua's HOLZF~\cite{obua-partizan-games}; the point is to have the closest possible integration
with the rest of Isabelle/HOL, minimising the amount of new notations and exploiting type classes.

There is a type \isa{V} of sets and a function
\isa{elts :: V\ {\isasymRightarrow}\ V\ set} mapping a set to its elements.
Classes simply have type \isa{V\ set}, and the predicate \isa{small} identifies those classes that correspond to actual sets.
Type classes connected with orders and lattices are used to minimise the amount of new notation
for concepts such as the subset relation, union and intersection.
Basic concepts are formalised: Cartesian products, disjoint sums, natural numbers, functions, etc.

More advanced set-theoretic concepts, such as transfinite induction, ordinals, cardinals
and the transitive closure of a set, are also provided.
The definition of addition and multiplication for general sets (not just ordinals) follows Kirby \cite{kirby-addition}.
The development includes essential results about cardinal arithmetic. It also develops ordinal exponentiation, Cantor normal form and the concept of indecomposable ordinals. There are numerous results about order types.

The theory provides two type classes with the aim of
facilitating developments that combine \isa{V} with other Isabelle/HOL types:
\isa{embeddable}, the class of types that can be injected into~\isa{V} (including \isa{V} itself as well as \isa{V*V}, \isa{V\ list}, etc.), and
\isa{small}, the class of types that correspond to some ZF set.
\end{abstract}

\newpage
\tableofcontents

% sane default for proof documents
\parindent 0pt\parskip 0.5ex

% generated text of all theories
\newpage
\input{session}

\section{Acknowledgements}
The author was supported by the ERC Advanced Grant ALEXANDRIA (Project 742178) funded by the European Research Council.

\bibliographystyle{abbrv}
\bibliography{root.bib}

\end{document}
