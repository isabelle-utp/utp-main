\documentclass[11pt,a4paper]{article}
\usepackage[utf8]{inputenc}
\usepackage{isabelle,isabellesym}

% further packages required for unusual symbols (see also
% isabellesym.sty), use only when needed

\usepackage{amssymb}
  %for \<leadsto>, \<box>, \<diamond>, \<sqsupset>, \<mho>, \<Join>,
  %\<lhd>, \<lesssim>, \<greatersim>, \<lessapprox>, \<greaterapprox>,
  %\<triangleq>, \<yen>, \<lozenge>

%\usepackage{eurosym}
  %for \<euro>

%\usepackage[only,bigsqcap]{stmaryrd}
  %for \<Sqinter>

%\usepackage{eufrak}
  %for \<AA> ... \<ZZ>, \<aa> ... \<zz> (also included in amssymb)

%\usepackage{textcomp}
  %for \<onequarter>, \<onehalf>, \<threequarters>, \<degree>, \<cent>,
  %\<currency>

% this should be the last package used
\usepackage{pdfsetup}

% urls in roman style, theory text in math-similar italics
\urlstyle{rm}
\isabellestyle{it}

% for uniform font size
%\renewcommand{\isastyle}{\isastyleminor}


\begin{document}

\title{Formalizing a Seligman-Style Tableau System for Hybrid Logic}
\author{Asta Halkjær From}
\maketitle

\begin{abstract}
This work is a formalization of soundness and completeness proofs
for a Seligman-style tableau system for hybrid logic. The completeness
result is obtained via a synthetic approach using maximally
consistent sets of tableau blocks. The formalization differs from
previous work~\cite{jlog17, aiml16} in a few ways. First, to avoid the need to backtrack in
the construction of a tableau, the formalized system has no unnamed
initial segment, and therefore no Name rule. Second, I show that the
full Bridge rule is admissible in the system. Third, I start from rules
restricted to only extend the branch with new formulas, including only
witnessing diamonds that are not already witnessed, and show that
the unrestricted rules are admissible. Similarly, I start from simpler
versions of the @-rules and show that these are sufficient.
The GoTo rule is restricted using a notion of potential such that each
application consumes potential and potential is earned through applications of
the remaining rules. I show that if a branch can be closed then it can
be closed starting from a single unit. Finally, Nom is restricted by
a fixed set of allowed nominals. The resulting system should be terminating.
\end{abstract}

\section*{Preamble}

The formalization was part of the author's MSc thesis in Computer Science and Engineering at the Technical University of Denmark (DTU).

\paragraph{Supervisors:}

\begin{itemize}
  \item Jørgen Villadsen
  \item Alexander Birch Jensen (co-supervisor)
  \item Patrick Blackburn (Roskilde University, external supervisor)
\end{itemize}

\newpage
\tableofcontents

\newpage

% sane default for proof documents
\parindent 0pt\parskip 0.5ex

% generated text of all theories
\input{session}

\nocite{*}

% optional bibliography
\bibliographystyle{abbrv}
\bibliography{root}

\addcontentsline{toc}{section}{References}

\end{document}

%%% Local Variables:
%%% mode: latex
%%% TeX-master: t
%%% End:
