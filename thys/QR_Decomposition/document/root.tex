\documentclass[11pt,a4paper]{article}
\usepackage{isabelle,isabellesym}
\usepackage{amssymb}

% this should be the last package used
\usepackage{pdfsetup}

% urls in roman style, theory text in math-similar italics
\urlstyle{rm}
\isabellestyle{it}


\begin{document}

\title{QR Decomposition}
\author{By Jose Divas\'on and Jes\'us Aransay\thanks{This research has been funded 
  by the research grant FPI-UR-12 of the Universidad de La Rioja and by the project MTM2014-54151-P from Ministerio de Econom\'ia y Competitividad
(Gobierno de Espa\~na).}}
\maketitle


\begin{abstract}
In this work we present a formalization of the QR decomposition, an algorithm which decomposes
a real matrix $A$ in the product of another two matrices $Q$ and $R$, where $Q$ is an orthogonal matrix
and $R$ is invertible and upper triangular. The algorithm is useful for the least squares problem, 
i.e. the computation of the best approximation of an unsolvable system of linear equations. 
As a side-product, the Gram-Schmidt process has also been formalized. A refinement using immutable arrays is presented as well. 
The development relies, among others, on the AFP entry \emph{Implementing field extensions of the form $\mathbb{Q}[\sqrt{b}]$} by Ren\'e Thiemann,
which allows to execute the algorithm using symbolic computations. Verified code can be generated and executed using floats as well.

\end{abstract}

\tableofcontents

% sane default for proof documents
\parindent 0pt\parskip 0.5ex

% generated text of all theories
\input{session}

\end{document}

%%% Local Variables:
%%% mode: latex
%%% TeX-master: t
%%% End:
