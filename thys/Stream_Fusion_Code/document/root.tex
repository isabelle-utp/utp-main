\documentclass[11pt,a4paper]{scrartcl}
\usepackage{isabelle,isabellesym}
\usepackage{amsmath,amssymb}
\usepackage{stmaryrd}

% this should be the last package used
\usepackage{pdfsetup}

% urls in roman style, theory text in math-similar italics
\urlstyle{rm}
\isabellestyle{it}

% for uniform font size
\renewcommand{\isastyle}{\isastyleminor}


\begin{document}

\title{Stream Fusion in HOL}
\author{Andreas Lochbihler \and Alexandra Maximova}
\maketitle

\begin{abstract}
  Stream Fusion is a system for removing intermediate list data structures from functional programs,
  in particular \href{http://hackage.haskell.org/package/stream-fusion}{Haskell}.
  This entry adapts stream fusion to Isabelle/HOL and its code generator.
  We define stream types for finite and possibly infinite lists and stream versions for most of the
  fusible list functions in the theories \isa{List} and \isa{Coinductive\isacharunderscore List},
  and prove them correct with respect to the conversion functions between lists and streams.
  The Stream Fusion transformation itself is implemented as a simproc in the preprocessor of the 
  code generator.

  Brian Huffman's AFP entry \cite{Huffman2009AFP} formalises stream fusion in HOLCF for the domain of lazy lists to prove the GHC compiler rewrite rules correct.
  In contrast, this work enables Isabelle's code generator to perform stream fusion itself.
  To that end, it covers both finite and coinductive lists from the HOL library and the Coinductive entry.
  The fusible list functions require specification and proof principles different from Huffman's.
\end{abstract}

\clearpage

\tableofcontents

\clearpage

% sane default for proof documents
\parindent 0pt\parskip 0.5ex

% generated text of all theories
\input{session}

%\section{Conclusion}\label{sec:concl}
  We have presented a verification of two variants of Gabow's algorithm: Computation of the strongly connected components of
  a graph, and emptiness check of a generalized B\"uchi automaton. We have extracted efficient code with a performance comparable to a
  reference implementation in Java.
  
  We have modularized the formalization in two directions: First, we share most of the proofs between the two variants of the algorithm. Second,
  we use a stepwise refinement approach to separate the algorithmic ideas and the correctness proof from implementation details.
  Sharing of the proofs reduced the overall effort of developing both algorithms. Using a stepwise refinement approach allowed us to
  formalize an efficient implementation, without making the correctness proof complex and unmanageable by cluttering it with implementation details.

  Our development approach is independent of Gabow's algorithm, and can be re-used for the verification of other algorithms.

  \paragraph{Current and Future Work} 
  An important direction of future work is to fine-tune the implementation of 
  the emptiness check algorithm for speed, as speed of the checking algorithm
  directely influences the performance of the modelchecker.


% optional bibliography
\bibliographystyle{abbrv}
\bibliography{root}

\end{document}
