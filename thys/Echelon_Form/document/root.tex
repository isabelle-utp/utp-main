\documentclass[11pt,a4paper]{article}
\usepackage{isabelle,isabellesym}
\usepackage{amssymb}

% this should be the last package used
\usepackage{pdfsetup}

% urls in roman style, theory text in math-similar italics
\urlstyle{rm}
\isabellestyle{it}


\begin{document}

\title{Echelon Form}
\author{By Jose Divas\'on and Jes\'us Aransay\thanks{This research has been funded 
  by the research grant FPI-UR-12 of the Universidad de La Rioja and by the project MTM2014-54151-P from Ministerio de Econom\'ia y Competitividad
(Gobierno de Espa\~na).}}
\maketitle

\begin{abstract}
In this work we present the formalization of an algorithm to compute the Echelon Form of
a matrix. We have proved its existence over Bezout domains and we have made it
executable over Euclidean domains, such as $\mathbb{Z}$ and $\mathbb{K}[x]$. This allows us
to compute determinants, inverses and characteristic polynomials of matrices.
The work is based on the \emph{HOL-Multivariate Analysis} library, and on both the Gauss-Jordan 
and Cayley-Hamilton AFP entries. As a by-product, some algebraic structures have been implemented (principal ideal domains, Bezout domains\dots).
The algorithm has been refined to immutable arrays and code can be generated to 
functional languages as well.
\end{abstract}

\tableofcontents

% sane default for proof documents
\parindent 0pt\parskip 0.5ex

% generated text of all theories
\input{session}

\end{document}

%%% Local Variables:
%%% mode: latex
%%% TeX-master: t
%%% End:
