\documentclass[11pt,a4paper]{article}
\usepackage{isabelle,isabellesym}
\usepackage{amsfonts, amsmath, amssymb}

% this should be the last package used
\usepackage{pdfsetup}

% urls in roman style, theory text in math-similar italics
\urlstyle{rm}
\isabellestyle{it}


\begin{document}

\title{The Prime Number Theorem}
\author{Manuel Eberl and Larry Paulson}
\maketitle

\begin{abstract}
This article provides a short proof of the Prime Number Theorem in several equivalent forms, most notably $\pi(x)\sim x / \ln x$ where $\pi(x)$ is the number of primes no larger than $x$. It also defines other basic number-theoretic functions related to primes like Chebyshev's $\vartheta$ and $\psi$ and the ``$n$-th prime number'' function $p_n$. We also show various bounds and relationship between these functions are shown. Lastly, we derive Mertens' First and Second Theorem, i.\,e.\ $\sum_{p\leq x} \frac{\ln p}{p} = \ln x + O(1)$ and $\sum_{p\leq x} \frac{1}{p} = \ln\ln x + M + O(1/\ln x)$. We also give explicit bounds for the remainder terms.

The proof of the Prime Number Theorem builds on a library of Dirichlet series and analytic combinatorics. We essentially follow the presentation by Newman~\cite{newman1998analytic}. The core part of the proof is a Tauberian theorem for Dirichlet series, which is proven using complex analysis and then used to strengthen Mertens' First Theorem to $\sum_{p\leq x} \frac{\ln p}{p} = \ln x + c + o(1)$.

A variant of this proof has been formalised before by Harrison in HOL Light~\cite{harrison-pnt}, and formalisations of Selberg's elementary proof exist both by Avigad \textit{et al.}\ \cite{avigad_pnt}\ in Isabelle and by Carneiro~\cite{carneiro_pnt} in Metamath.
The advantage of the analytic proof is that, while it requires more powerful mathematical tools, it is considerably shorter and clearer. This article attempts to provide a short and clear formalisation of all components of that proof using the full range of mathematical machinery available in Isabelle, staying as close as possible to Newman's simple paper proof.
\end{abstract}

\newpage
\tableofcontents
\newpage
\parindent 0pt\parskip 0.5ex

\input{session}

\section{Acknowledgements}
Paulson was supported by the ERC Advanced Grant ALEXANDRIA (Project 742178) funded by the European Research Council at the University of Cambridge, UK.

\bibliographystyle{abbrv}
\bibliography{root}

\end{document}

%%% Local Variables:
%%% mode: latex
%%% TeX-master: t
%%% End:
