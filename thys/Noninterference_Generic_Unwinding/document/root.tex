\documentclass[11pt,a4paper]{article}
\usepackage{isabelle,isabellesym}
\renewcommand{\isastyletxt}{\isastyletext}

% this should be the last package used
\usepackage{pdfsetup}

% urls in roman style, theory text in math-similar italics
\urlstyle{rm}
\isabellestyle{it}


\begin{document}

\title{The Generic Unwinding Theorem\\for CSP Noninterference Security}
\author{Pasquale Noce\\Security Certification Specialist at Arjo Systems - Gep S.p.A.\\pasquale dot noce dot lavoro at gmail dot com\\pasquale dot noce at arjowiggins-it dot com}
\maketitle

\begin{abstract}
The classical definition of noninterference security for a deterministic state
machine with outputs requires to consider the outputs produced by machine
actions after any trace, i.e. any indefinitely long sequence of actions, of the
machine. In order to render the verification of the security of such a machine
more straightforward, there is a need of some sufficient condition for security
such that just individual actions, rather than unbounded sequences of actions,
have to be considered.

By extending previous results applying to transitive noninterference policies,
Rushby has proven an unwinding theorem that provides a sufficient condition of
this kind in the general case of a possibly intransitive policy. This condition
has to be satisfied by a generic function mapping security domains into
equivalence relations over machine states.

An analogous problem arises for CSP noninterference security, whose definition
requires to consider any possible future, i.e. any indefinitely long sequence of
subsequent events and any indefinitely large set of refused events associated to
that sequence, for each process trace.

This paper provides a sufficient condition for CSP noninterference security,
which indeed requires to just consider individual accepted and refused events
and applies to the general case of a possibly intransitive policy. This
condition follows Rushby's one for classical noninterference security, and has
to be satisfied by a generic function mapping security domains into equivalence
relations over process traces; hence its name, Generic Unwinding Theorem.
Variants of this theorem applying to deterministic processes and trace set
processes are also proven. Finally, the sufficient condition for security
expressed by the theorem is shown not to be a necessary condition as well, viz.
there exists a secure process such that no domain-relation map satisfying the
condition exists.
\end{abstract}

\tableofcontents

% sane default for proof documents
\parindent 0pt\parskip 0.5ex

% generated text of all theories
\input{session}

% bibliography
\bibliographystyle{abbrv}
\bibliography{root}

\end{document}
