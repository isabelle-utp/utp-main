\documentclass[11pt,a4paper]{report}
\usepackage{isabelle,isabellesym}
\usepackage[only,bigsqcap]{stmaryrd}  % for \<Sqinter>
\usepackage[ngerman]{babel}  % for guillemots

% this should be the last package used
\usepackage{pdfsetup}

% urls in roman style, theory text in math-similar italics
\urlstyle{rm}
\isabellestyle{it}


\begin{document}

\title{G\"odel's Incompleteness Theorems}
\author{Lawrence C. Paulson}
\maketitle

\begin{abstract}
G\"odel's two incompleteness theorems \cite{goedel-I} are formalised, following a careful presentation by {\'S}wierczkowski \cite{swierczkowski-finite}, in the theory of hereditarily finite sets. This represents the first ever machine-assisted proof of the second incompleteness theorem. Compared with traditional formalisations using Peano arithmetic \cite{boolos-provability}, coding is simpler, with no need to formalise the notion of multiplication (let alone that of a prime number) in the formalised calculus upon which the theorem is based.
However, other technical problems had to be solved in order to complete the argument.
\end{abstract}

\tableofcontents

% include generated text of all theories
\input{session}

\bibliographystyle{abbrv}
\bibliography{root}

\end{document}
