\documentclass[11pt,a4paper,DIV=11]{scrartcl}
\usepackage[utf8]{inputenc}
\usepackage[T1]{fontenc}
\usepackage{isabelle,isabellesym}
\usepackage{amsmath,amsfonts}
\usepackage[standard]{ntheorem}

\renewcommand{\bf}{\normalfont\bfseries}
\renewcommand{\rm}{\normalfont\rmfamily}
\renewcommand{\it}{\normalfont\itshape}

\usepackage{pdfsetup}

\hypersetup{
  pdfinfo={
    Title={Menger's Theorem},
    Subject={},
    Keywords={Formal Proof, Graph Theory, Menger's Theorem},
    Author={Christoph Dittmann},
    Creator={}
  },
  bookmarksopen=true,
  bookmarksnumbered,
  bookmarksopenlevel=2,
  bookmarksdepth=3
}

% urls in roman style, theory text in math-similar italics
\urlstyle{rm}
\isabellestyle{it}

% for uniform font size
%\renewcommand{\isastyle}{\isastyleminor}

\begin{document}

\title{Menger's Theorem}
\author{Christoph Dittmann\\isabelle@christoph-d.de}
\date{\today}
\maketitle

\begin{abstract}
  We present a formalization of Menger's Theorem for directed and
  undirected graphs in Isabelle/HOL.  This well-known result shows
  that if two non-adjacent distinct vertices $u,v$ in a directed graph
  have no separator smaller than $n$, then there exist $n$ internally
  vertex-disjoint paths from $u$ to $v$.

  The version for undirected graphs follows immediately because
  undirected graphs are a special case of directed graphs.
\end{abstract}

\tableofcontents
\newpage

\section{Introduction}

Given two non-adjacent distinct vertices $u, v$ in a finite directed
graph, a \emph{$u$-$v$-separator} is a set of vertices $S$ with $u
\notin S, v \notin S$ such that every $u$-$v$-path visits a vertex of
$S$.  Two $u$-$v$-paths are \emph{internally vertex-disjoint} if their
intersection is exactly $\{u,v\}$.

A famous classical result of graph theory relates the size of a
minimum separator to the maximal number of internally vertex-disjoint
paths.

\begin{theorem}[Menger \cite{Menger1927}]\label{thm:menger}
  Let $u,v$ be two non-adjacent distinct vertices. Then the size of a
  minimum $u$-$v$-separator equals the maximal number of pairwise
  internally vertex-disjoint $u$-$v$-paths.
\end{theorem}

This theorem has many proofs, but as far as the author is aware, there
was no formalized proof.  We follow a proof given by William McCuaig,
who calls it ``A simple proof of Menger's
theorem''~\cite{DBLP:journals/jgt/McCuaig84}.  His proof is roughly
one page in length.  Our formalization is significantly longer than
that because we had to fill in a lot of details.

Most of the work goes into showing the following theorem, which proves
one direction of Theorem~\ref{thm:menger}.

\begin{theorem}
  Let $u,v$ be two non-adjacent distinct vertices.  If every
  $u$-$v$-separator has size at least $n$, then there exists $n$
  pairwise internally vertex-disjoint $u$-$v$-paths.
\end{theorem}

Compared to this, the other direction of Theorem~\ref{thm:menger} is
easy because the existence of $n$ internally vertex-disjoint paths
implies that every separator needs to cut at least these paths, so
every separator needs to have size at least $n$.

\section{Relation to Min-Cut Max-Flow}

Another famous result of graph theory is the Min-Cut Max-Flow Theorem,
stating that the size of a minimum $u$-$v$-cut equals the value of a
maximum $u$-$v$-flow.  There exists a formalization of a very general
version of this theorem for countable graphs in the Archive of Formal
Proofs, written by Andreas Lochbihler~\cite{MFMC_Countable-AFP}.

Technically, our version of Menger's Theorem should follow from
Lochbihler's very general result.  However, the author was of the
opinion that a fresh formalization of Menger's Theorem was warranted
given the complexity of the Min-Cut Max-Flow formalization.  Our
formalization is about a sixth of the size of the Min-Cut Max-Flow
formalization (not counting comments).  It may also be easier to grasp
by readers who are unfamiliar with the intricacies of countable
networks.

Let us also note that the Min-Cut Max-Flow Theorem considers
\emph{edge cuts} whereas Menger's Theorem works with \emph{vertex
  cuts}.  This is a minor difference because one can be reduced to the
other, but it makes Menger's Theorem not a trivial corollary of the
Min-Cut Max-Flow formalization.


% sane default for proof documents
\parindent 0pt\parskip 0.5ex

% generated text of all theories
\input{session}

\clearpage
\phantomsection
\addcontentsline{toc}{section}{Bibliography}
\bibliographystyle{alphaurl}
\bibliography{root}

\end{document}

%%% Local Variables:
%%% mode: latex
%%% TeX-master: t
%%% End:
