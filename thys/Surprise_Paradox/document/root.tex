\documentclass[11pt,a4paper]{article}
\usepackage{isabelle,isabellesym}

\usepackage[utf8]{inputenc}
% further packages required for unusual symbols (see also
% isabellesym.sty), use only when needed

\usepackage[ngerman]{babel}  % for guillemots

%\usepackage{amssymb}
  %for \<leadsto>, \<box>, \<diamond>, \<sqsupset>, \<mho>, \<Join>,
  %\<lhd>, \<lesssim>, \<greatersim>, \<lessapprox>, \<greaterapprox>,
  %\<triangleq>, \<yen>, \<lozenge>

%\usepackage{eurosym}
  %for \<euro>

%\usepackage[only,bigsqcap]{stmaryrd}
  %for \<Sqinter>

%\usepackage{eufrak}
  %for \<AA> ... \<ZZ>, \<aa> ... \<zz> (also included in amssymb)

%\usepackage{textcomp}
  %for \<onequarter>, \<onehalf>, \<threequarters>, \<degree>, \<cent>,
  %\<currency>

% this should be the last package used
\usepackage{pdfsetup}

% urls in roman style, theory text in math-similar italics
\urlstyle{rm}
\isabellestyle{it}

\begin{document}

\title{The Surprise Paradox}
\author{Joachim Breitner\\
Programming Paradigms Group\\
Karlsruhe Institute for Technology\\
\url{breitner@kit.edu}}

\maketitle

\begin{abstract}
In 1964, Fitch showed that the paradox of the surprise hanging can be resolved by showing that the judge’s verdict is inconsistent. His formalization builds on Gödel’s coding of provability.

In this theory, we reproduce his proof in Isabelle, building on Paulson’s formalisation of Gödel’s incompleteness theorems.
\end{abstract}

\tableofcontents

\bigskip

% sane default for proof documents
\parindent 0pt\parskip 0.5ex

% generated text of all theories
\input{session}

% optional bibliography
\bibliographystyle{abbrv}
\bibliography{root}

\end{document}

%%% Local Variables:
%%% mode: latex
%%% TeX-master: t
%%% End:
