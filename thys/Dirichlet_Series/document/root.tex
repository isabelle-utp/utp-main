\documentclass[11pt,a4paper]{article}
\usepackage{isabelle,isabellesym}
\usepackage{amsfonts, amsmath, amssymb}

% this should be the last package used
\usepackage{pdfsetup}

% urls in roman style, theory text in math-similar italics
\urlstyle{rm}
\isabellestyle{it}


\begin{document}

\title{Dirichlet Series}
\author{Manuel Eberl}
\maketitle

\begin{abstract}
This entry is a formalisation of much of Chapters 2, 3, and 11 of Apostol's ``Introduction to Analytic Number Theory''~\cite{apostol}. This includes:
\begin{itemize} 
	\item Definitions and basic properties for several number-theoretic functions (Euler's $\varphi$, M\"{o}bius $\mu$, Liouville's $\lambda$, the divisor function $\sigma$, von Mangoldt's $\Lambda$)
\item Executable code for most of these functions, the most efficient implementations using the factoring algorithm by Thiemann\ \emph{et al.}
\item Dirichlet products and formal Dirichlet series
\item Analytic results connecting convergent formal Dirichlet series to complex functions
\item Euler product expansions
\item Asymptotic estimates of number-theoretic functions including the density of squarefree integers and the average number of divisors of a natural number
\end{itemize}
These results are useful as a basis for developing more number-theoretic results, such as the Prime Number Theorem.
\end{abstract}

\newpage
\tableofcontents
\newpage
\parindent 0pt\parskip 0.5ex

\input{session}

\bibliographystyle{abbrv}
\bibliography{root}

\end{document}

%%% Local Variables:
%%% mode: latex
%%% TeX-master: t
%%% End:
