\documentclass[11pt,a4paper]{article}
\usepackage{isabelle,isabellesym}

% this should be the last package used
\usepackage{pdfsetup}

% urls in roman style, theory text in math-similar italics
\urlstyle{rm}
\isabellestyle{it}


\begin{document}

\title{Gauss-Jordan algorithm and its applications}
\author{By Jose Divas\'on and Jes\'us Aransay\thanks{This research has been funded by 
  the research grant FPIUR12 of the Universidad de La Rioja.}}
\maketitle


\begin{abstract}
In this contribution, we present a formalization of the well-known Gauss-Jordan algorithm.
It states that any matrix over a field can be transformed by means of elementary row operations to a
matrix in reduced row echelon form. The formalization is based on the
Rank Nullity Theorem entry of the AFP and on the HOL-Multivariate-Analysis session of Isabelle, 
where matrices are represented as functions over finite types. We have set up properly 
the code generator to make this representation executable. In order to improve the 
performance, a refinement to immutable arrays has been carried out. We have formalized some of the applications
of the Gauss-Jordan algorithm. Thanks to this development, the 
following facts can be computed over matrices whose elements belong to a field:

\begin{itemize}
 \item Ranks
 \item Determinants
 \item Inverses
 \item Bases and dimensions of the null space, left null space, column space and row space of a matrix
 \item Solutions of systems of linear equations (considering any case, 
   including systems with one solution, multiple solutions and with no solution)
\end{itemize}

Code can be exported to both SML and Haskell. In addition, we have introduced some serializations 
(for instance, from \emph{bit} in Isabelle to booleans in SML and Haskell, 
and from \emph{rat} in Isabelle to the corresponding one in Haskell), that speed up the performance.

\end{abstract}

\tableofcontents

% sane default for proof documents
\parindent 0pt\parskip 0.5ex

% generated text of all theories
\input{session}

% optional bibliography
\bibliographystyle{abbrv}
\bibliography{root}

\end{document}

%%% Local Variables:
%%% mode: latex
%%% TeX-master: t
%%% End:
