\documentclass[11pt,a4paper]{article}
\usepackage{isabelle,isabellesym}
\renewcommand{\isastyletxt}{\isastyletext}

% further packages required for unusual symbols (see also
% isabellesym.sty), use only when needed

%\usepackage{amssymb}
  %for \<leadsto>, \<box>, \<diamond>, \<sqsupset>, \<mho>, \<Join>,
  %\<lhd>, \<lesssim>, \<greatersim>, \<lessapprox>, \<greaterapprox>,
  %\<triangleq>, \<yen>, \<lozenge>

%\usepackage{eurosym}
  %for \<euro>

%\usepackage[only,bigsqcap]{stmaryrd}
  %for \<Sqinter>

%\usepackage{eufrak}
  %for \<AA> ... \<ZZ>, \<aa> ... \<zz> (also included in amssymb)

%\usepackage{textcomp}
  %for \<onequarter>, \<onehalf>, \<threequarters>, \<degree>, \<cent>,
  %\<currency>

% this should be the last package used
\usepackage{pdfsetup}

% urls in roman style, theory text in math-similar italics
\urlstyle{rm}
\isabellestyle{it}

% for uniform font size
%\renewcommand{\isastyle}{\isastyleminor}


\begin{document}

\title{Verification of a Diffie-Hellman Password-based Authentication Protocol by Extending the Inductive Method}
\author{Pasquale Noce\\Security Certification Specialist at Arjo Systems, Italy\\pasquale dot noce dot lavoro at gmail dot com\\pasquale dot noce at arjosystems dot com}
\maketitle

\begin{abstract}
This paper constructs a formal model of a Diffie-Hellman password-based
authentication protocol between a user and a smart card, and proves its
security. The protocol provides for the dispatch of the user's password to the
smart card on a secure messaging channel established by means of Password
Authenticated Connection Establishment (PACE), where the mapping method being
used is Chip Authentication Mapping. By applying and suitably extending
Paulson's Inductive Method, this paper proves that the protocol establishes
trustworthy secure messaging channels, preserves the secrecy of users'
passwords, and provides an effective mutual authentication service. What is
more, these security properties turn out to hold independently of the secrecy of
the PACE authentication key.
\end{abstract}

\tableofcontents

% sane default for proof documents
\parindent 0pt\parskip 0.5ex

% generated text of all theories
\input{session}

% bibliography
\bibliographystyle{abbrv}
\bibliography{root}

\end{document}
