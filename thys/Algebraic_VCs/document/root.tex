\documentclass[11pt,a4paper]{article}
\usepackage{isabelle,isabellesym}

% further packages required for unusual symbols (see also
% isabellesym.sty), use only when needed

%\usepackage{amssymb}
  %for \<leadsto>, \<box>, \<diamond>, \<sqsupset>, \<mho>, \<Join>,
  %\<lhd>, \<lesssim>, \<greatersim>, \<lessapprox>, \<greaterapprox>,
  %\<triangleq>, \<yen>, \<lozenge>

%\usepackage{eurosym}
  %for \<euro>

%\usepackage[only,bigsqcap]{stmaryrd}
  %for \<Sqinter>

%\usepackage{eufrak}
  %for \<AA> ... \<ZZ>, \<aa> ... \<zz> (also included in amssymb)

%\usepackage{textcomp}
  %for \<onequarter>, \<onehalf>, \<threequarters>, \<degree>, \<cent>,
  %\<currency>

% this should be the last package used
\usepackage{pdfsetup}

% urls in roman style, theory text in math-similar italics
\urlstyle{rm}
\isabellestyle{it}

% for uniform font size
%\renewcommand{\isastyle}{\isastyleminor}


\begin{document}

\title{Program Construction and Verification Components Based on
  Kleene Algebra}
\author{Victor B. F. Gomes and Georg Struth}
\maketitle

\begin{abstract}
  Variants of Kleene algebra support program construction and
  verification by algebraic reasoning. This entry provides a
  verification component for Hoare logic based on Kleene algebra with
  tests, verification components for weakest preconditions and
  strongest postconditions based on Kleene algebra with domain and a
  component for step-wise refinement based on refinement Kleene
  algebra with tests. In addition to these components for the partial
  correctness of while programs, a verification component for total
  correctness based on divergence Kleene algebras and one for
  (partial correctness) of recursive programs based on domain
  quantales are provided. Finally we have integrated memory models for
  programs with pointers and a program trace semantics into the
  weakest precondition component.
\end{abstract}

\tableofcontents

% sane default for proof documents
\parindent 0pt\parskip 0.5ex

\section{Introductory Remarks}

These Isabelle theories provide program construction and verification
components for simpe while programs based on variants of Kleene
algebra with tests and Kleene algebra with domain, as well as a
component for parameterless recursive programs based on domain
quantales. The general approach consists in using the algebras for
deriving verification conditions for the control flow of
programs. They are linked by formal soundness proofs with denotational
program semantics of the store and data domain---here predominantly
with a relational semantics.  Assignment laws can then be derived in
this semantics. Program construction and verification tasks are
performed within the concrete semantics as well; structured syntax for
programs could easily be added, but is not provided at the moment.

All components are correct by construction relative to Isabelle's
small trustworthy core, as our soundness proofs make the axiomatic
extensions provided by the algebras consistent with respect to it.

The main components  are integrated into previous AFP entries
for Kleene algebras~\cite{afp:ka}, Kleene algebras with
tests~\cite{afp:kat} and Kleene algebras with domain~\cite{afp:kad}.
As an overview and perhaps for educational purposes, we have also
added two standalone components based on Hoare logic and weakest
(liberal) preconditions  that use only Isabelle's main libraries.

Background information on the general approach and the first main
component, which is based on Kleene algebra with tests, can be found
in~\cite{ArmstrongGS15}. An introduction to Kleene algebra with domain
is given in~\cite{DesharnaisS11}; a paper describing the corresponding
verification component in detail is in preparation.

We are planning to add further components and expand and restructure
the existing ones in the future. We would like to invite anyone
interested in the algebraic approach to collaborate with us on these
and contribute to this project.

% generated text of all theories
\input{session}

% optional bibliography
\bibliographystyle{abbrv}
\bibliography{root}

\end{document}

%%% Local Variables:
%%% mode: latex
%%% TeX-master: t
%%% End:
