\documentclass[11pt,a4paper]{article}
\usepackage{isabelle,isabellesym,amsmath,amssymb,a4wide}
\usepackage{graphicx,xcolor}

\newcommand{\imp}{\rightarrow}
\newcommand{\biimp}{\leftrightarrow}
\newcommand{\all}{\forall}
\newcommand{\ex}{\exists}
\newcommand{\seq}{\vdash}
\newcommand{\nec}{\Box} % necessarily
\newcommand{\pos}{\Diamond} % possibly
\newcommand{\ess}[2]{#1 \ \mathit{ess.} \ #2}
\newcommand{\NE}{\mathit{NE}}

% this should be the last package used
\usepackage{pdfsetup}

% urls in roman style, theory text in math-similar italics
\urlstyle{rm}
\isabellestyle{it}


\begin{document}

\title{G\"odel's God in Isabelle/HOL}
\author{Christoph Benzm\"uller and Bruno Woltzenlogel Paleo}
%\date{November 1, 2013}
\maketitle

%\noindent\colorbox{gray}{\includegraphics[width=.99\textwidth]{$HOME/GoedelGod/Talks/FU-Berlin/ScottsScriptGrab}} %$

\begin{figure}[h]
\noindent\fcolorbox{gray}{white}{
\begin{minipage}{.96\textwidth}\small
\begin{itemize}
\item[A1] Either a property or its negation is positive, but not
  both:  \hfill 
  $\all \phi [P(\neg \phi) \biimp \neg P(\phi)]$ \\[-1.5em]
\item[A2] A property necessarily implied \\ by a
  positive property is positive: \phantom{b} \hfill 
  $\all \phi \all \psi [(P(\phi) \wedge \nec \all x [\phi(x)
  \imp \psi(x)]) \imp P(\psi)]$ \\[-1.5em]
\item[T1] Positive properties are possibly exemplified: \hfill $\all
  \phi [P(\phi) \imp \pos \ex x \phi(x)]$ \\[-1.5em]
\item[D1] A \emph{God-like} being possesses all positive properties: \hfill
  $G(x) \biimp \forall \phi [P(\phi) \to \phi(x)]$ \\[-1.5em]
\item[A3]  The property of being God-like is positive: \hfill   $P(G)$ \\[-1.5em]
\item[C\phantom{1}] Possibly, God exists: \hfill $\pos \ex x G(x)$ \\[-1.5em]
\item[A4]  Positive properties are necessarily positive: \hfill 
  $\all \phi [P(\phi) \to \Box \; P(\phi)]$ \\[-1.5em]
\item[D2] An \emph{essence} of an individual is a property possessed by it \\ and necessarily implying any of its properties: \\
  \phantom{b} \hfill $\ess{\phi}{x} \biimp \phi(x) \wedge \all
  \psi (\psi(x) \imp \nec \all y (\phi(y) \imp \psi(y)))$ \\[-1.5em]
\item[T2]  Being God-like is an essence of any
  God-like being: \hfill $\all x [G(x) \imp \ess{G}{x}]$ \\[-1.5em]
\item[D3] \emph{Necessary existence} of an individual is \\ the necessary exemplification of all its essences: 
  \phantom{b} \hfill $\NE(x) \biimp \all \phi [\ess{\phi}{x} \imp \nec
  \ex y \phi(y)]$ \\[-1.5em]
\item[A5] Necessary existence is a positive property: \hfill $P(\NE)$ \\[-1.5em]
\item[T3] Necessarily, God exists: \hfill $\nec \ex x G(x)$ 
\end{itemize}
\end{minipage}
}
\caption{Scott's version of G\"odel's ontological argument \cite{ScottNotes}.} 
\end{figure}
\vskip1em

%\tableofcontents

% sane default for proof documents
\parindent 0pt\parskip 0.5ex

% generated text of all theories
\input{session}

\paragraph{Acknowledgments:} Nik Sultana, Jasmin Blanchette and Larry Paulson provided 
very important help on issues related to consistency checking in Isabelle. Jasmin Blanchette instructed us on 
producing Isabelle sessions and he showed us some useful tricks in Isabelle.

%\small
\begin{thebibliography}{10}

\bibitem{B9}
C.~Benzm{\"u}ller and L.C. Paulson.
\newblock Exploring properties of normal multimodal logics in simple type
  theory with {LEO-II}.
\newblock In {\em {Festschrift in Honor of {Peter B. Andrews} on His 70th
  Birthday}}, pp. 386--406. College Publications.

\bibitem{J23}
C.~Benzm{\"u}ller and L.C. Paulson.
\newblock Quantified multimodal logics in simple type theory.
\newblock {\em Logica Universalis (Special Issue on Multimodal Logics)},
  7(1):7--20, 2013.

\bibitem{LEO-II}
C.~Benzm{\"u}ller, F.~Theiss, L.~Paulson, and A.~Fietzke.
\newblock {LEO-II} - a cooperative automatic theorem prover for higher-order
  logic.
\newblock In {\em Proc. of IJCAR 2008}, volume 5195 of {\em LNAI}, pp.
  162--170. Springer, 2008.

\bibitem{Coq}
Y.~Bertot and P.~Casteran.
\newblock {\em {Interactive Theorem Proving and Program Development}}.
\newblock Springer, 2004.

\bibitem{Sledgehammer}
J.C. Blanchette, S.~B\"ohme, and L.C. Paulson.
\newblock Extending {Sledgehammer} with {SMT} solvers.
\newblock {\em Journal of Automated Reasoning}, 51(1):109--128, 2013.

\bibitem{Nitpick}
J.C. Blanchette and T.~Nipkow.
\newblock Nitpick: A counterexample generator for higher-order logic based on a
  relational model finder.
\newblock In {\em Proc. of ITP 2010}, LNCS 6172, pp. 131--146.
  Springer, 2010.

\bibitem{Satallax}
C.E. Brown.
\newblock Satallax: An automated higher-order prover.
\newblock In {\em Proc. of IJCAR 2012}, LNAI 7364, pp. 111 -- 117.
  Springer, 2012.

\bibitem{GoedelNotes}
K.~G\"odel.
\newblock {\em Appendix A. Notes in Kurt G\"odel's Hand}, pp. 144--145.
\newblock In  \cite{sobel2004logic}, 2004.


\bibitem{Metis}
J.~Hurd.
\newblock First-order proof tactics in higher-order logic theorem provers.
\newblock In {\em Design and Application of Strategies/Tactics in Higher Order
  Logics, NASA Tech. Rep. NASA/CP-2003-212448}, 2003.

\bibitem{Isabelle}
T.~Nipkow, L.C. Paulson, and M.~Wenzel.
\newblock {\em {Isabelle/HOL: A Proof Assistant for Higher-Order Logic}}.
\newblock LNCS 2283. Springer, 2002.

\bibitem{rushby}
J.~Rushby.
\newblock The Ontological Argument in PVS. 
\newblock {\em CAV Workshop ``Fun With Formal Methods'}, St. Petersburg, Russia, 13th of July 2013.

\bibitem{ScottNotes}
D.~Scott.
\newblock {\em Appendix B. Notes in Dana Scott's Hand}, pp. 145--146.
\newblock In  \cite{sobel2004logic}, 2004.

\bibitem{sobel2004logic}
J.H. Sobel.
\newblock {\em Logic and Theism: Arguments for and Against Beliefs in God}.
\newblock Cambridge University Press, 2004.

\bibitem{J22}
G.~Sutcliffe and C.~Benzm{\"u}ller.
\newblock Automated reasoning in higher-order logic using the {TPTP THF}
  infrastructure.
\newblock {\em Journal of Formalized Reasoning}, 3(1):1--27, 2010.


\end{thebibliography}

\end{document}

%%% Local Variables:
%%% mode: latex
%%% TeX-master: t
%%% End:
