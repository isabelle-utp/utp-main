\documentclass[11pt,a4paper]{article}
\usepackage{isabelle,isabellesym}

% further packages required for unusual symbols (see also
% isabellesym.sty), use only when needed

%\usepackage{amssymb}
  %for \<leadsto>, \<box>, \<diamond>, \<sqsupset>, \<mho>, \<Join>,
  %\<lhd>, \<lesssim>, \<greatersim>, \<lessapprox>, \<greaterapprox>,
  %\<triangleq>, \<yen>, \<lozenge>

%\usepackage{eurosym}
  %for \<euro>

%\usepackage[only,bigsqcap]{stmaryrd}
  %for \<Sqinter>

%\usepackage{eufrak}
  %for \<AA> ... \<ZZ>, \<aa> ... \<zz> (also included in amssymb)

%\usepackage{textcomp}
  %for \<onequarter>, \<onehalf>, \<threequarters>, \<degree>, \<cent>,
  %\<currency>

% this should be the last package used
\usepackage{pdfsetup}

% urls in roman style, theory text in math-similar italics
\urlstyle{rm}
\isabellestyle{it}

% for uniform font size
%\renewcommand{\isastyle}{\isastyleminor}


\begin{document}


\title{Attack Trees in Isabelle for GDPR compliance of IoT healthcare systems}
\author{Florian Kamm\"uller}

\maketitle

\begin{abstract}
In this article, we present a proof theory for Attack Trees. Attack Trees are a well established and 
useful model for the construction of attacks on systems since they allow a stepwise exploration of 
high level attacks in application scenarios. Using the expressiveness of Higher Order Logic in Isabelle, 
we succeed in developing a generic theory of Attack Trees with a state-based semantics based on Kripke 
structures and CTL (see \cite{kam:16b} for more details). 
The resulting framework allows mechanically supported logic analysis of the meta-theory 
of the proof calculus of Attack Trees and at the same time the developed proof theory enables application 
to case studies. 
A central correctness and completeness result proved in Isabelle establishes a connection 
between the notion of Attack tTree validity and CTL. 
The application is illustrated on the example of a healthcare IoT system and GDPR compliance verification.
A more detailed account of the Attack Tree formalisation is given in \cite{kam:18b} and the case study 
is described in detail in \cite{kam:18a}.
%bla \cite{kk:16}\cite{kp:16}\cite{mw:09}\cite{kk:20}
\end{abstract}
\tableofcontents

% sane default for proof documents
\parindent 0pt\parskip 0.5ex

% generated text of all theories
\input{session}

% optional bibliography
\bibliographystyle{abbrv}
\bibliography{root}


\end{document}

%%% Local Variables:
%%% mode: latex
%%% TeX-master: t
%%% End:
