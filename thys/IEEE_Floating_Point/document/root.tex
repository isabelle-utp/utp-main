\documentclass[11pt,a4paper]{article}
\usepackage{isabelle,isabellesym}

% this should be the last package used
\usepackage{pdfsetup}

% urls in roman style, theory text in math-similar italics
\urlstyle{rm}
\isabellestyle{it}

\parindent 0pt\parskip 0.5ex


\begin{document}

\title{A Formal Model of IEEE Floating Point Arithmetic}
\author{Lei Yu}
\maketitle

\begin{abstract}
  This development provides a formal model of IEEE-754 floating-point
  arithmetic. This formalization, including formal specification of the
  standard and proofs of important properties of floating-point arithmetic,
  forms the foundation for verifying programs with floating-point computation.
  There is also a code generation setup for floats so that we can execute
  programs using this formalization in functional programming languages. The
  definitions of the IEEE standard in Isabelle is ported from HOL Light
  \cite{harrison1997floating}.
\end{abstract}

\tableofcontents

% include generated text of all theories
\input{session}

\bibliographystyle{abbrv}
\bibliography{root}

\end{document}
