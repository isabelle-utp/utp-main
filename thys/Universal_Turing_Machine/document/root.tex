\documentclass{article}
%\documentclass[runningheads]{llncs}
\usepackage{isabelle}
\usepackage{isabellesym}
\usepackage{times}
\usepackage{amssymb}
\usepackage{amsmath}
\usepackage{stmaryrd}
\usepackage{mathpartir}
%\usepackage{pdfsetup}
\usepackage{tikz}
\usepackage{pgf}
\usepackage{color}
\usetikzlibrary{calc}
\usetikzlibrary{positioning}


%% for testing
%\usepackage{endnotes}
%\let\footnote=\endnote
\def\inst#1{\unskip$^{#1}$}

% urls in roman style, theory text in math-similar italics
\isabellestyle{it}

% this should be the last package used
\usepackage{pdfsetup}


% gray boxes
\definecolor{mygrey}{rgb}{.80,.80,.80}

% mathpatir
\mprset{sep=0.9em}
\mprset{center=false}
\mprset{flushleft=true}

% for uniform font size
%\renewcommand{\isastyle}{\isastyleminor}

\def\dn{\,\stackrel{\mbox{\scriptsize def}}{=}\,}
\renewcommand{\isasymequiv}{$\dn$}
\renewcommand{\isasymemptyset}{$\varnothing$}
\renewcommand{\isacharunderscore}{\mbox{$\_$}}
\renewcommand{\isasymiota}{}
\newcommand{\isasymulcorner}{$\ulcorner$}
\newcommand{\isasymurcorner}{$\urcorner$}
\newcommand{\chapter}{\section}

\begin{document}


\title{Universal Turing Machine and Computability Theory in Isabelle/HOL}
\author{Jian Xu\inst{2} \and Xingyuan Zhang\inst{2} \and Christian Urban\inst{1} \and Sebastiaan J. C. Joosten\inst{3} \vspace{3pt} \\
\inst{1}King's College London, UK \\ \inst{2}PLA University of Science and Technology, China \\ \inst{3}University of Twente, the Netherlands}

\maketitle


\begin{abstract}
We formalise results from computability theory: recursive functions, undecidability of the halting problem, and the existence of a universal Turing machine. 
This formalisation is the AFP entry corresponding to: Mechanising Turing Machines and Computability Theory in Isabelle/HOL, ITP 2013
\end{abstract}

The AFP entry and by extension this document is largely written by Jian Xu, Xingyuan Zhang, and Christian Urban.
The Universal Turing Machine is well explained in this document, starting at Figure~\ref{prepare_input}.
Regardless, you may want to read the original ITP article~\cite{Xu13} instead of this pdf document corresponding to the AFP entry.
If you are just interested in results about Turing Machines and Computability theory: the main book used for this formalisation is by Boolos~\cite{Boolos87}.

Sebastiaan J. C. Joosten contributed mainly by making the files ready for the AFP.
The need for a good formalisation of Turing Machines arose from realising that the current formalisation of saturation graphs~\cite{Graph_Saturation-AFP} is missing a key undecidability result present in the original paper~\cite{Joosten18}.
Recently, an undecidability result has been added to the AFP by Bertram Felgenhauer~\cite{Minsky_Machines-AFP}, using a definition of computably enumerable sets formalised by Michael Nedzelsky~\cite{Recursion-Theory-I-AFP}.
Showing the equivalence of these entirely separate notions of computability and decidability remains future work.

% generated text of all theories
\input{session}

% optional bibliography
\bibliographystyle{abbrv}
\bibliography{root}

\end{document}

%%% Local Variables:
%%% mode: latex
%%% TeX-master: t
%%% End:
