\documentclass[11pt,a4paper]{article}
\usepackage{isabelle,isabellesym}
\usepackage{fullpage}
\usepackage[usenames,dvipsnames]{color}
\usepackage{graphicx}
\usepackage{document}

% further packages required for unusual symbols (see also
% isabellesym.sty), use only when needed

\usepackage{amssymb}
  %for \<leadsto>, \<box>, \<diamond>, \<sqsupset>, \<mho>, \<Join>,
  %\<lhd>, \<lesssim>, \<greatersim>, \<lessapprox>, \<greaterapprox>,
  %\<triangleq>, \<yen>, \<lozenge>

\usepackage[greek,english]{babel}
  %option greek for \<euro>
  %option english (default language) for \<guillemotleft>, \<guillemotright>

\usepackage[only,bigsqcap]{stmaryrd}
  %for \<Sqinter>

\usepackage{eufrak}
  %for \<AA> ... \<ZZ>, \<aa> ... \<zz> (also included in amssymb)

%\usepackage{textcomp}
  %for \<onequarter>, \<onehalf>, \<threequarters>, \<degree>, \<cent>,
  %\<currency>

\usepackage{stmaryrd}

% this should be the last package used
\usepackage{pdfsetup}

% urls in roman style, theory text in math-similar italics
\urlstyle{rm}
\isabellestyle{it}

% for uniform font size
%\renewcommand{\isastyle}{\isastyleminor}

% Lens operations

\newcommand{\view}{\mathit{V}}
\newcommand{\src}{\mathit{S}}
\newcommand{\lsbs}{{L}}
\newcommand{\lput}{\textit{\textsf{put}}}
\newcommand{\lget}{\textit{\textsf{get}}}
\newcommand{\lcreate}{\mathit{create}}
\newcommand{\lto}{\Longrightarrow}
\newcommand{\lsubseteq}{\preceq}
\newcommand{\lsupseteq}{\mathop{\supseteq_\lsbs}}
\newcommand{\lequiv}{\approx}
\newcommand{\lcomp}{;_\lsbs}
\newcommand{\lplus}{+_\lsbs}
\newcommand{\lquot}{\mathop{/\!_\lsbs}}
\newcommand{\lindep}{\mathop{\,\bowtie\,}}
\newcommand{\lone}{\mathbf{1}}
\newcommand{\lzero}{\mathbf{0}}
\newcommand{\lfst}{\textit{\textsf{\textbf{fst}}}}
\newcommand{\lsnd}{\textit{\textsf{\textbf{snd}}}}

\setcounter{topnumber}{1}
\setcounter{bottomnumber}{1}
\setcounter{totalnumber}{1}

\begin{document}

\title{Optics in Isabelle/HOL}

\author{Simon Foster and Frank Zeyda \\[.5ex] University of York, UK \\[2ex] \texttt{\small $\{$simon.foster,frank.zeyda$\}$@york.ac.uk}}

\maketitle

\begin{abstract}
  Lenses provide an abstract interface for manipulating data types through spatially-separated views. They are defined
  abstractly in terms of two functions, $\lget$, the return a value from the source type, and $\lput$ that updates
  the value. We mechanise the underlying theory of lenses, in terms of an algebraic hierarchy of lenses, including
  well-behaved and very well-behaved lenses, each lens class being characterised by a set of lens laws. We also mechanise 
  a lens algebra in Isabelle that enables their composition and comparison, so as to allow construction of complex lenses.
  This is accompanied by a large library of algebraic laws. Moreover we also show how the lens classes can be applied by 
  instantiating them with a number of Isabelle data types. This theory development is based on our recent 
  papers~\cite{Foster16a,Foster2020-IsabelleUTP}, which show how lenses can be used to unify heterogeneous representations 
  of state-spaces in formalised programs.
\end{abstract}

\tableofcontents

% sane default for proof documents
\parindent 0pt\parskip 0.5ex

% generated text of all theories
\input{session}

\vspace{4ex}

% Acknowledgments 
\noindent\textbf{Acknowledgements}. This work is partly supported by EU H2020 project \emph{INTO-CPS}, grant agreement
644047. \url{http://into-cps.au.dk/}. We would also like to thank Prof. Burkhart Wolff and Dr. Achim Brucker
for their generous and helpful comments on our work, and particurlarly their invaluable advice on Isabelle
mechanisation and ML coding.

% optional bibliography
\bibliographystyle{abbrv}
\bibliography{root}

\end{document}
