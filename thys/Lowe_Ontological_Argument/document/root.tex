\documentclass[11pt,a4paper]{article}
\usepackage{authblk}
%\usepackage{a4wide}
\usepackage{isabelle,isabellesym}

% further packages required for unusual symbols (see also
% isabellesym.sty), use only when needed

\usepackage{amssymb}
  %for \<leadsto>, \<box>, \<diamond>, \<sqsupset>, \<mho>, \<Join>,
  %\<lhd>, \<lesssim>, \<greatersim>, \<lessapprox>, \<greaterapprox>,
  %\<triangleq>, \<yen>, \<lozenge>

%\usepackage{eurosym}
  %for \<euro>

%\usepackage[only,bigsqcap]{stmaryrd}
  %for \<Sqinter>

%\usepackage{eufrak}
  %for \<AA> ... \<ZZ>, \<aa> ... \<zz> (also included in amssymb)

%\usepackage{textcomp}
  %for \<onequarter>, \<onehalf>, \<threequarters>, \<degree>, \<cent>,
  %\<currency>

% this should be the last package used
\usepackage{pdfsetup}

% urls in roman style, theory text in math-similar italics
\urlstyle{rm}
\isabellestyle{it}

% for uniform font size
%\renewcommand{\isastyle}{\isastyleminor}


\begin{document}

\title{Computer-assisted Reconstruction and Assessment\\
	of E. J. Lowe's Modal Ontological Argument}
\author[1]{David Fuenmayor}
\author[2,1]{Christoph Benzm\"uller}
\affil[1]{Freie Universit\"at Berlin, Germany}
\affil[2]{University of Luxembourg, Luxembourg}

\maketitle

\begin{abstract}
	Computers may help us to understand --not just verify-- philosophical arguments.
	By utilizing modern proof assistants in an iterative interpretive process, we can
	reconstruct and assess an argument by fully formal means.
	Through the mechanization of a variant of St. Anselm's ontological argument by E. J. Lowe,
	which is a paradigmatic example of a natural-language argument with strong ties to metaphysics and religion,
	we offer an ideal showcase for our computer-assisted interpretive method.
\end{abstract}

\tableofcontents

% sane default for proof documents
\parindent 0pt\parskip 0.5ex

% generated text of all theories
\input{session}

% optional bibliography
\bibliographystyle{abbrv}
\bibliography{root}

\end{document}

%%% Local Variables:
%%% mode: latex
%%% TeX-master: t
%%% End:
