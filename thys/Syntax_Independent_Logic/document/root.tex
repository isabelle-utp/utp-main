\documentclass[10pt,a4paper]{report}
\usepackage{isabelle,isabellesym}

\usepackage{a4wide}
\usepackage[english]{babel}
\usepackage{eufrak}
\usepackage{amssymb}

% this should be the last package used
\usepackage{pdfsetup}

% urls in roman style, theory text in math-similar italics
\urlstyle{rm}
\isabellestyle{literal}


\begin{document}

\title{Syntax-Independent Logic Infrastructure}
\author{Andrei Popescu \and Dmitriy Traytel}

\maketitle

\begin{abstract} We formalize a notion of logic whose terms and formulas are kept abstract. In particular,
logical connectives, substitution, free variables, and provability are not defined, but characterized by
their general properties as locale assumptions. Based on this abstract characterization, we develop further
reusable reasoning infrastructure. For example, we define parallel substitution (along with proving its
characterizing theorems) from single-point substitution. Similarly, we develop a natural
deduction style proof system starting from the abstract Hilbert-style one. These one-time efforts benefit
different concrete logics satisfying our locales' assumptions.

We instantiate the syntax-independent logic infrastructure to Robinson arithmetic (also known as Q) in the
AFP entry \href{https://www.isa-afp.org/entries/Robinson_Arithmetic.html}{Robinson\_Arithmetic} and to
hereditarily finite set theory in the AFP entries
\href{https://www.isa-afp.org/entries/Goedel_HFSet_Semantic.html}{Goedel\_HFSet\_Semantic} and
\href{https://www.isa-afp.org/entries/Goedel_HFSet_Semanticless.html}{Goedel\_HFSet\_Semanticless}, which are
part of our formalization of G\"odel's Incompleteness Theorems described in our CADE-27
paper~\cite{DBLP:conf/cade/0001T19}. \end{abstract}

\tableofcontents

% sane default for proof documents
\parindent 0pt\parskip 0.5ex

% generated text of all theories
\input{session}

\bibliographystyle{abbrv}
\bibliography{root}

\end{document}

%%% Local Variables:
%%% mode: latex
%%% TeX-master: t
%%% End:
