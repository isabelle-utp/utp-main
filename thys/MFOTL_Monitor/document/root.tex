\documentclass[10pt,a4paper]{article}
\usepackage{isabelle,isabellesym}

\usepackage{a4wide}
\usepackage[english]{babel}
\usepackage{eufrak}
\usepackage{amssymb}

% this should be the last package used
\usepackage{pdfsetup}

% urls in roman style, theory text in math-similar italics
\urlstyle{rm}
\isabellestyle{literal}


\begin{document}

\title{Formalization of a Monitoring Algorithm for\\ Metric First-Order Temporal Logic}
\author{Joshua Schneider \and Dmitriy Traytel}

\maketitle

\begin{abstract} A monitor is a runtime verification tool that solves the following
problem: Given a stream of time-stamped events and a policy formulated in a specification
language, decide whether the policy is satisfied at every point in the stream. We verify
the correctness of an executable monitor for specifications given as formulas in metric
first-order temporal logic (MFOTL)~\cite{BasinKMZ-JACM15}, an expressive extension of
linear temporal logic with real-time constraints and
first-order quantification. The verified monitor implements a simplified variant of the algorithm
used in the efficient MonPoly monitoring tool~\cite{monpoly}. The formalization is presented in a
RV 2019 paper~\cite{SchneiderBKT-RV19}, which also compares the output of the
verified monitor to that of other monitoring tools on randomly generated inputs. This
case study revealed several errors in the optimized but unverified tools. \end{abstract}

\tableofcontents

% sane default for proof documents
\parindent 0pt\parskip 0.5ex

% generated text of all theories
\input{session}

\bibliographystyle{abbrv}
\bibliography{root}

\end{document}

%%% Local Variables:
%%% mode: latex
%%% TeX-master: t
%%% End:
