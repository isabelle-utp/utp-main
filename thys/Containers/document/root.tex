\documentclass[11pt,a4paper]{book}
\usepackage{isabelle,isabellesym}
\usepackage{amssymb}
\usepackage[english]{babel}
\usepackage[only,bigsqcap]{stmaryrd}
\usepackage{booktabs}

% this should be the last package used
\usepackage{pdfsetup}

% urls in roman style, theory text in math-similar italics
\urlstyle{rm}
\isabellestyle{it}

% Tweaks
\newcounter{TTStweak_tag}
\setcounter{TTStweak_tag}{0}
\newcommand{\setTTS}{\setcounter{TTStweak_tag}{1}}
\newcommand{\resetTTS}{\setcounter{TTStweak_tag}{0}}
\newcommand{\insertTTS}{\ifnum\value{TTStweak_tag}=1 \ \ \ \fi}

\renewcommand{\isakeyword}[1]{\resetTTS\emph{\bf\def\isachardot{.}\def\isacharunderscore{\isacharunderscorekeyword}\def\isacharbraceleft{\{}\def\isacharbraceright{\}}#1}}
\renewcommand{\isachardoublequoteopen}{\insertTTS}
\renewcommand{\isachardoublequoteclose}{\setTTS}
\renewcommand{\isanewline}{\mbox{}\par\mbox{}\resetTTS}

\renewcommand{\isamarkupcmt}[1]{\hangindent5ex{\isastylecmt --- #1}}

\newcommand{\isaheader}[1]{\section{#1}}

\makeatletter
\newenvironment{abstract}{%
  \small
  \begin{center}%
    {\bfseries \abstractname\vspace{-.5em}\vspace{\z@}}%
  \end{center}%
  \quotation}{\endquotation}
\makeatother

\begin{document}

\title{Light-Weight Containers}
\author{Andreas Lochbihler}
\maketitle

\begin{abstract}
  This development provides a framework for container types like sets and maps such that generated code implements these containers with different (efficient) data structures.
  Thanks to type classes and refinement during code generation, this light-weight approach can seamlessly replace Isabelle's default setup for code generation.
  Heuristics automatically pick one of the available data structures depending on the type of elements to be stored, but users can also choose on their own.
  The extensible design permits to add more implementations at any time.

  To support arbitrary nesting of sets, we define a linear order on sets based on a linear order of the elements and provide efficient implementations.
  It even allows to compare complements with non-complements.
\end{abstract}

\clearpage

\tableofcontents

\clearpage

% sane default for proof documents
\parindent 0pt\parskip 0.5ex

\chapter{Introduction}

This development focuses on generating efficient code for container types like sets and maps.
It falls into two parts: 
First, we define linear order on sets (Ch.~\ref{chapter:linear:order:set}) that is efficiently executable given a linear order on the elements.
Second, we define an extensible framework LC (for light-weight containers) that supports multiple (efficient) implementations of container types (Ch.~\ref{chapter:light-weight:containers}) in generated code.
Both parts heavily exploit type classes and the refinement features of the code generator \cite{HaftmannKrausKuncarNipkow2013ITP}.
This way, we are able to implement the HOL types for sets and maps directly, as the name light-weight containers (LC) emphasises.

In comparison with the Isabelle Collections Framework (ICF) \cite{LammichLochbihler2010ITP,Lammich2009AFP}, the style of refinement is the major difference.
In the ICF, the container types are replaced with the types of the data structures inside the logic.
Typically, the user has to define his operations that involve maps and sets a second time such that they work on the concrete data structures; then, she has to prove that both definitions agree.
With LC, the refinement happens inside the code generator.
Hence, the formalisation can stick with the types $'a\ set$ and $('a, 'b)\ mapping$ and there is no need to duplicate definitions or prove refinement.
The drawback is that with LC, we can only implement operations that can be fully specified on the abstract container type.
In particular, the internal representation of the implementations may not affect the result of the operations.
For example, it is not possible to pick non-deterministically an element from a set or fold a set with a non-commutative operation, i.e., the result depends on the order of visiting the elements.

For more documentation and introductory material refer to the userguide (Chapter~\ref{chapter:Userguide}) and the ITP-2013 paper \cite{Lochbihler2013ITP}.

% generated text of all theories
\input{session}

%\section{Conclusion}\label{sec:concl}
  We have presented a verification of two variants of Gabow's algorithm: Computation of the strongly connected components of
  a graph, and emptiness check of a generalized B\"uchi automaton. We have extracted efficient code with a performance comparable to a
  reference implementation in Java.
  
  We have modularized the formalization in two directions: First, we share most of the proofs between the two variants of the algorithm. Second,
  we use a stepwise refinement approach to separate the algorithmic ideas and the correctness proof from implementation details.
  Sharing of the proofs reduced the overall effort of developing both algorithms. Using a stepwise refinement approach allowed us to
  formalize an efficient implementation, without making the correctness proof complex and unmanageable by cluttering it with implementation details.

  Our development approach is independent of Gabow's algorithm, and can be re-used for the verification of other algorithms.

  \paragraph{Current and Future Work} 
  An important direction of future work is to fine-tune the implementation of 
  the emptiness check algorithm for speed, as speed of the checking algorithm
  directely influences the performance of the modelchecker.


% optional bibliography
\bibliographystyle{abbrv}
\bibliography{root}

\end{document}

%%% Local Variables:
%%% mode: latex
%%% TeX-master: t
%%% End:
