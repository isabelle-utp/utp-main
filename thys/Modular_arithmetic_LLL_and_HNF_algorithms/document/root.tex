\documentclass[11pt,a4paper]{article}
\usepackage{isabelle,isabellesym}
	
% further packages required for unusual symbols (see also
% isabellesym.sty), use only when needed

%\usepackage{amssymb}
  %for \<leadsto>, \<box>, \<diamond>, \<sqsupset>, \<mho>, \<Join>,
  %\<lhd>, \<lesssim>, \<greatersim>, \<lessapprox>, \<greaterapprox>,
  %\<triangleq>, \<yen>, \<lozenge>

%\usepackage{eurosym}
  %for \<euro>

%\usepackage[only,bigsqcap]{stmaryrd}
  %for \<Sqinter>

%\usepackage{eufrak}
  %for \<AA> ... \<ZZ>, \<aa> ... \<zz> (also included in amssymb)

%\usepackage{textcomp}
  %for \<onequarter>, \<onehalf>, \<threequarters>, \<degree>, \<cent>,
  %\<currency>

% this should be the last package used
\usepackage{pdfsetup}

% urls in roman style, theory text in math-similar italics
\urlstyle{rm}
\isabellestyle{it}

% for uniform font size
%\renewcommand{\isastyle}{\isastyleminor}


\begin{document}

\title{Two algorithms based on modular arithmetic: lattice basis reduction and Hermite normal form computation\footnote{Supported
by FWF (Austrian Science Fund) project Y757 
and by project MTM2017-88804-P (Spanish Ministry of Science and Innovation).}}
%\title{Modulo arithmetic-based algorithms for lattice basis reduction and for computing the Hermite normal form}
\author{Ralph Bottesch \and Jose Divas\'on \and Ren\'e Thiemann}
\maketitle

\begin{abstract}
We verify two algorithms for which modular arithmetic plays an essential role: Storjohann's variant of the LLL lattice basis reduction algorithm and Kopparty's algorithm for computing the Hermite normal form of a matrix. To do this, we also formalize some facts about the modulo operation with symmetric range. Our implementations are based on the original papers, but are otherwise efficient. For basis reduction we formalize two versions: one that includes all of the optimizations/heuristics from Storjohann's paper, and one excluding a heuristic that we observed to often decrease efficiency. We also provide a fast, self-contained certifier for basis reduction, based on the efficient Hermite normal form algorithm.
\end{abstract}

\tableofcontents

% sane default for proof documents
\parindent 0pt\parskip 0.5ex

% generated text of all theories
\input{session}

% optional bibliography
%\bibliographystyle{abbrv}
%\bibliography{root}

\end{document}

%%% Local Variables:
%%% mode: latex
%%% TeX-master: t
%%% End:
