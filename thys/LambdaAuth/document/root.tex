\documentclass[10pt,a4paper]{article}
\usepackage{isabelle,isabellesym}

\usepackage{a4wide}
\usepackage[english]{babel}
\usepackage{eufrak}

% this should be the last package used
\usepackage{pdfsetup}

% urls in roman style, theory text in math-similar italics
\urlstyle{rm}
\isabellestyle{literal}


\begin{document}

\title{Formalization of Generic Authenticated Data Structures}
\author{Matthias Brun \and Dmitriy Traytel}

\maketitle

\begin{abstract} Authenticated data structures are a technique for outsourcing data storage and maintenance to an untrusted server.
The server is required to produce an efficiently checkable and cryptographically secure proof that it carried out precisely the
requested computation. Miller et al.~\cite{adsg} introduced $\lambda\bullet$ (pronounced \emph{lambda auth})---a functional
programming language with a built-in primitive authentication construct, which supports a wide range of user-specified authenticated
data structures while guaranteeing certain correctness and security properties for all well-typed programs.
%
We formalize $\lambda\bullet$ and prove its correctness and security properties. With Isabelle's help, we uncover and repair several
mistakes in the informal proofs and lemma statements. Our findings are summarized in a paper draft~\cite{gadsf}.
\end{abstract}

\tableofcontents

% sane default for proof documents
\parindent 0pt\parskip 0.5ex

% generated text of all theories
\input{session}

\bibliographystyle{abbrv}
\bibliography{root}

\end{document}

%%% Local Variables:
%%% mode: latex
%%% TeX-master: t
%%% End:
