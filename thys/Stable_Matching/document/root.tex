\documentclass[11pt,a4paper]{article}
\usepackage[a4paper,margin=1cm,footskip=.5cm]{geometry}
\usepackage{isabelle,isabellesym}
\usepackage{amsmath}
\usepackage[only,bigsqcap]{stmaryrd}

\usepackage[utf8]{inputenc}

% Bibliography
\usepackage[authoryear,sort]{natbib}
\bibpunct();A{},

% Allow pdflatex to do some fancier spacing.
\usepackage{microtype}

% this should be the last package used
\usepackage{pdfsetup}

% urls in roman style, theory text in math-similar italics
\urlstyle{rm}
\isabellestyle{it}

% for uniform font size
\renewcommand{\isastyle}{\isastyleminor}

\begin{document}

% sane default for proof documents
\parindent 0pt\parskip 0.5ex

\title{Stable Matching}
\author{Peter Gammie}
\maketitle

\begin{abstract}
  \noindent We mechanize proofs of several results from the
  \emph{matching with contracts} literature, which generalize those of
  the classical two-sided matching scenarios that go by the name of
  \emph{stable marriage}. Our focus is on game theoretic issues. Along
  the way we develop executable algorithms for computing optimal
  stable matches.
\end{abstract}

\tableofcontents


\section{Introduction}
\label{sec:introduction}

As economists have turned their attention to the design of such
markets as school enrolments, internships, and housing refugees
\citep{AnderssonEhlers:2016}, particular \emph{matching} scenarios
have proven to be useful models. \citet{Roth:2015} defines matching as
``economist-speak for how we get the many things we choose in life
that also must choose us,'' and one such two-sided market is now
colloquially known as the
\href{https://en.wikipedia.org/wiki/Stable_marriage_problem}{\emph{stable
    marriage problem}}.  It was initially investigated by
\citet{GaleShapley:1962}, who introduced the key solution concept of
\emph{stability}, and the \emph{deferred-acceptance algorithm} that
efficiently constructs stable matches for it. We refer readers
unfamiliar with this classical work to \S\ref{sec:sotomayor}, where we
formalize this scenario and mechanize a non-constructive existence
proof of stable matches due to \citet{Sotomayor:1996}. Further
in-depth treatment can be found in the very readable monographs by
\citet{GusfieldIrving:1989} (algorithmics), \citet{RothSotomayor:1990}
(economics), and \citet{Manlove:2013}.

Recently \citet{HatfieldMilgrom:2005} (see also
\citet{Fleiner:2000,Fleiner:2002,Fleiner:2003}) have recast the
two-sided matching model to incorporate \emph{contracts}, which
intuitively allow agents to additionally indicate preferences over
conditions such as salary. By allowing many-to-one matches, some
aspects of a labour market can be modelled. Their analysis leans
heavily on the lattice structure of the stable matches, and yields
pleasingly simple and general algorithms
(\S\ref{sec:contracts}). Later work trades this structure for
generality, and the analysis becomes more intricate
(\S\ref{sec:cop}). The key game-theoretic result is the (one-sided)
strategy-proofness of the optimal stable match
(\S\ref{sec:strategic}).

This work was motivated by the difficulty of navigating the literature
on \emph{matching with contracts} by non-specialists, as observed by
\citet{VCG-EC:2015,VCG-AFP:2015}. We impose some order by formalizing
much of it in Isabelle/HOL \citep{Nipkow-Paulson-Wenzel:2002}, a proof
assistant for a simply-typed higher-order logic. By carefully writing
definitions that are executable and testable, we avail ourselves of
Isabelle's automatic tools, specifically \verb!nitpick! and
\verb!sledgehammer!, to rapidly identify errors when formulating
assertions. We focus primarily on strategic (game theoretic) issues,
but our development is also intended to serve as a foundation for
further results.

The proof assistant forces us to take care of all details, which
yields a verbosity that may deter some readers. We suggest that most
will fare best by reading the definitions and
\isa{\isacommand{lemma}}/\isa{\isacommand{theorem}} statements
closely, and skipping the proofs. (The important results are labelled
\isa{\isacommand{theorem}} and \isa{\isacommand{proposition}}, but
often the \isa{\isacommand{lemma}}s contain the meat.) The material
in \S\ref{sec:cf} on choice functions is mostly for reference.

This PDF is generated directly from the development's sources and is
extensively hyperlinked, but for some purposes there is no substitute
to firing up Isabelle.


% generated text of all theories
\input{session}


\section{Concluding remarks}

We conclude with a brief and inexhaustive survey of related work.

\subsection{Related work}

\paragraph{Computer-assisted and formal reasoning.}
\citet{Bijlsma:1991} gives a formal pencil-and-paper derivation of the
Gale-Shapley deferred-acceptance algorithm under total strict
preferences and one-to-one matching (colloquially, a marriage
market). He provides termination and complexity arguments, and
discusses representation issues. \citet{HamidCastleberry:2010} treat
the same algorithm in the Coq proof assistant, give a termination
proof and show that it always yields a stable match. Both focus more
on reasoning about programs than the theory of stable
matches. Intriguingly, the latter claims that Akamai uses (modified)
stable matching to assign clients to servers in their content
distribution network.

\citet{DBLP:conf/atal/BrandtG14} use SAT technology to find results in
social choice theory. They claim that the encodings used by general
purpose tools like \verb!nitpick!  are too inefficient for their
application.

\paragraph{Stable matching.} In addition to the monographs
\citet{GusfieldIrving:1989,RothSotomayor:1990,Manlove:2013},
\citet{Roth:2008} provides a good overview up to 2007 of open problems
and other aspects of this topic that we did not explore here.
\citet{SonmezSwitzer:2013} incorporate quotas and put the COP to work
at the United States Military Academy. \citet{AnderssonEhlers:2016}
analyze the possibility of matching of refugees with landlords in
Sweden (without mentioning matching with contracts).

One of the more famous applications of matching theory is to kidney
donation \citep{Roth:2015}, a \emph{repugnant market} where the
economists' basic tool of pricing things is considered verboten. These
markets are sometimes, but not always, two-sided -- kidneys are often
exchanged due to compatibility issues, but there are also altruistic
donations and recipients who cannot reciprocate -- and so the model we
discussed here is not applicable. Instead generalizations of Gale's
\emph{top trading cycles} algorithm are pressed into service
\citep{ShapleyScarf:1974,AbdulkadirogluSonmez:1999,SonmezUnver:2010}.
Much recent work has hybridized these approaches -- for instance,
\citet{Dworczak:2016} uses a combination to enumerate all stable
matches.

\citet{Echenique:2012} shows that the matching with contracts model of
\S\ref{sec:contracts} is no more general than that of
\citet{KelsoCrawford:1982} (a job matching market with
salaries). \citeauthor{Schlegel:2015} generalizes this result to the
COP setting of \S\ref{sec:cop}, and moreover shows how lattice
structure can be recovered there, which yields a hospital-proposing
deferred-acceptance algorithm that relies only on unilaterally
substitutable hospital choice functions. See
\citet{HatfieldKominers:2016} for a discussion of the many-to-many
case.

\citet[Theorem~2.33]{RothSotomayor:1990} point to alternatives to the
deferred-acceptance algorithm, and to more general matching scenarios
involving couples and roommates. \citet{Manlove:2013} provides a
comprehensive survey of matching with preferences.

\paragraph{Further results: COP.} \citet{Afacan:2014} explores the
following two properties:
\begin{quote}

  \emph{[Population monotonicity]} says that no doctor is to be worse
  off whenever some others leave the market. \emph{[Resource
    monotonicity]}, on the other hand, requires that no doctor should
  lose whenever hospitals start hiring more doctors.

\end{quote}
He shows that the COP is population and resource monotonic under
\emph{irc} and \emph{bilateral\_substitutes}. Also \citet{Afacan:2015}
characterizes the COP by the properties \emph{truncation proof} (``no
doctor can ever benefit from truncating his preferences'') and
\emph{invariant to lower tail preferences change} (``any doctor's
assignment does not depend on his preferences over worse contracts'');
that the COP satisfies these properties was demonstrated in
\S\ref{sec:cop}. See also \citet{HatfieldKominersWestkamp:2016} for
another set of conditions that characterize the COP.

\citet{HirataKasuya:2016} show how the strategic results can be
obtained without the rural hospitals theorem, in a setting that
requires \emph{irc} but not substitutability.

\paragraph{Further results: Strategy.} There are many different ways
to think about the manipulation of economic mechanisms. Some continue
in the game-theoretic tradition \citep{Gonczarowski:2014}, and, for
instance, compare the manipulability of mechanisms that yield stable
matches \citep{ChenEgesdalPyciaUenmez:2016}. Techniques from computer
science help refine the notion of strategy-proofness
\citep{AshlagiGonczarowski:2015} and enable complexity-theoretic
arguments
\citep{DBLP:conf/atal/AzizSW15,DengShenTang:2016}. \citet{KojimaPathak:2009}
have analyzed the scope for manipulation in large matching markets.


\section{Acknowledgements}

I thank \href{http://www.dcs.gla.ac.uk/~rwi/}{Rob Irving} for a copy
of his excellent monograph \citep{GusfieldIrving:1989}, Jasmin
C. Blanchette for help with nitpick, Andreas Lochbihler for his
Bourbaki-Witt Fixpoint theory, Orhan Aygün for a helpful discussion of
\citet{AygunSonmez:2012-WP}, and Roman Werpachowski for general
comments.

\bibliographystyle{plainnat}
\bibliography{root}
\addcontentsline{toc}{section}{References}

\end{document}

%%% Local Variables:
%%% mode: latex
%%% TeX-master: t
%%% End:
