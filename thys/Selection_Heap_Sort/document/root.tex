\documentclass[11pt,a4paper]{article}
\usepackage{isabelle,isabellesym}

% this should be the last package used
\usepackage{pdfsetup}

% urls in roman style, theory text in math-similar italics
\urlstyle{rm}
\isabellestyle{it}


\begin{document}

\title{ Verification of Selection and Heap Sort Using Locales}
\author{Danijela Petrovi\'c}
\maketitle

\begin{abstract}
  Stepwise program refinement techniques can be used to simplify
  program verification. Programs are better understood since their
  main properties are clearly stated, and verification of rather
  complex algorithms is reduced to proving simple statements
  connecting successive program specifications. Additionally, it is
  easy to analyze similar algorithms and to compare their properties
  within a single formalization. Usually, formal analysis is not done
  in educational setting due to complexity of verification and a lack
  of tools and procedures to make comparison easy. Verification of an
  algorithm should not only give correctness proof, but also better
  understanding of an algorithm. If the verification is based on small
  step program refinement, it can become simple enough to be
  demonstrated within the university-level computer science
  curriculum. In this paper we demonstrate this and give a formal
  analysis of two well known algorithms (Selection Sort and Heap Sort)
  using proof assistant Isabelle/HOL and program refinement
  techniques.
\end{abstract}

\tableofcontents

% ------------------------------------------------------------------------------
\section{Introduction}
% ------------------------------------------------------------------------------

\paragraph*{Using program verification within computer science
  education.}
Program verification is usually considered to be too hard and long
process that acquires good mathematical background. A verification of
a program is performed using mathematical logic. Having the
specification of an algorithm inside the logic, its correctness can be
proved again by using the standard mathematical apparatus (mainly
induction and equational reasoning). These proofs are commonly complex
and the reader must have some knowledge about mathematical logic. The
reader must be familiar with notions such as satisfiability, validity,
logical consequence, etc. Any misunderstanding leads into a loss of
accuracy of the verification. These formalizations have common
disadvantage, they are too complex to be understood by students, and
this discourage students most of the time. Therefore, programmers and
their educators rather use traditional (usually trial-and-error)
methods.

However, many authors claim that nowadays education lacks the formal
approach and it is clear why many advocate in using proof
assistants\cite{LSDtrip}. This is also the case with computer science
education. Students are presented many algorithms, but without formal
analysis, often omitting to mention when algorithm would not work
properly. Frequently, the center of a study is implementation of an
algorithm whereas understanding of its structure and its properties is
put aside. Software verification can bring more formal approach into
teaching of algorithms and can have some advantages over traditional
teaching methods.
\begin{itemize}
\item Verification helps to point out what are the requirements and
  conditions that an algorithm satisfies (pre-conditions,
  post-conditions and invariant conditions) and then to apply this
  knowledge during programming. This would help both students and
  educators to better understand input and output specification and
  the relations between them.
\item Though program works in general case, it can happen that it does
  not work for some inputs and students must be able to detect these
  situations and to create software that works properly for all
  inputs.
\item It is suitable to separate abstract algorithm from its specific
  implementation. Students can compare properties of different
  implementations of the same algorithms, to see the benefits of one
  approach or another. Also, it is possible to compare different
  algorithms for same purpose (for example, for searching element,
  sorting, etc.) and this could help in overall understanding of
  algorithm construction techniques.
\end{itemize}
Therefore, lessons learned from formal verification of an algorithm
can improve someones style of programming.

\paragraph*{Modularity and refinement.}
The most used languages today are those who can easily be compiled
into efficient code. Using heuristics and different data types makes
code more complex and seems to novices like perplex mixture of many
new notions, definitions, concepts. These techniques and methods in
programming makes programs more efficient but are rather hard to be
intuitively understood. On the other hand highly accepted principle in
nowadays programming is modularity. Adhering to this principle enables
programmer to easily maintain the code.

The best way to apply modularity on program verification and to make
verification flexible enough to add new capabilities to the program
keeping current verification intact is \emph{program
  refinement}. Program refinement is the verifiable transformation of
an abstract (high-level) formal specification into a concrete
(low-level) executable program. It starts from the abstract level,
describing only the requirements for input and output. Implementation
is obtained at the end of the verification process (often by means of
code generation \cite{codegeneration}). Stepwise refinement allows this process to
be done in stages. There are many benefits of using refinement
techniques in verification.

\begin{itemize}
\item It gives a better understanding of programs that are verified.
\item The algorithm can be analyzed and understood on different level
  of abstraction.
\item It is possible to verify different implementations for some part
  of the program, discussing the benefits of one approach or another.
\item It can be easily proved that these different implementation
  share some same properties which are proved before splitting into
  two directions.
\item It is easy to maintain the code and the verification. Usually,
  whenever the implementation of the program changes, the correctness
  proofs must be adapted to these changes, and if refinement is used,
  it is not necessary to rewrite entire verification, just add or
  change small part of it.
\item Using refinement approach makes algorithm suitable for a case
  study in teaching. Properties and specifications of the program are
  clearly stated and it helps teachers and students better to teach or
  understand them.
\end{itemize}

We claim that the full potential of refinement comes only when it is
applied stepwise, and in many small steps. If the program is refined
in many steps, and data structures and algorithms are introduced
one-by-one, then proving the correctness between the successive
specifications becomes easy. Abstracting and separating each
algorithmic idea and each data-structure that is used to give an
efficient implementation of an algorithm is very important task in
programmer education.

As an example of using small step refinement, in this paper we analyze
two widely known algorithms, Selection Sort and Heap Sort. There are
many reasons why we decided to use them.
\begin{itemize}
\item They are largely studied in different contexts and they are
  studied in almost all computer science curricula.
\item They belong to the same family of algorithms and they are good
  example for illustrating the refinement techniques. They are a nice
  example of how one can improve on a same idea by introducing more
  efficient underlying data-structures and more efficient algorithms.
\item Their implementation uses different programming constructs:
  loops (or recursion), arrays (or lists), trees, etc. We show how to
  analyze all these constructs in a formal setting.
\end{itemize}

There are many formalizations of sorting algorithms that are done both
automatically or interactively and they undoubtedly proved that these
algorithms are correct. In this paper we are giving a new approach in
their verification, that insists on formally analyzing connections
between them, instead of only proving their correctness (which has
been well established many times). Our central motivation is that
these connections contribute to deeper algorithm understanding much
more than separate verification of each algorithm.


% sane default for proof documents
\parindent 0pt\parskip 0.5ex

% generated text of all theories
\input{session}


% ------------------------------------------------------------------------------
\section{Related work}
\label{sec:related}
% ------------------------------------------------------------------------------

To study sorting algorithms from a top down was proposed in
\cite{meritt}. All sorting algorithms are based on divide-and-conquer
algorithm and all sorts are divided into two groups:
hard\_split/easy\_join and easy\_split/hard\_join. Fallowing this idea
in \cite{dps}, authors described sorting algorithms using
object-oriented approach. They suggested that this approach could be
used in computer science education and that presenting sorting
algorithms from top down will help students to understand them better.

The paper \cite{sortMorp} represent different recursion
patterns --- catamorphism, anamorphism, hylomorphism and
paramorphisms. Selection, buble, merge, heap and quick sort are
expressed using these patterns of recursion and it is shown that there
is a little freedom left in implementation level. Also, connection
between different patterns are given and thus a conclusion about
connection between sorting algorithms can be easily
conducted. Furthermore, in the paper are generalized tree data types
-- list, binary trees and binary leaf trees.

Satisfiability procedures for working with arrays was proposed in
paper ``What is decidable about arrays?''\cite{arrays}. This procedure
is called $SAT_A$ and can give an answer if two arrays are equal or if
array is sorted and so on. Completeness and soundness for procedures
are proved. There are, though, several cases when procedures are
unsatisfiable. They also studied theory of maps. One of the
application for these procedures is verification of sorting algorithms
and they gave an example that insertion sort returns sorted array. 

Tools for program verification are developed by different groups and
with different results. Some of them are automated and some are
half-automated.

Ralph-Johan Back and Johannes Eriksson \cite{socos} developed SOCOS,
tool for program verification based on invariant diagrams. SOCOS
environment supports interactive and non-interactive checking of
program correctness. For each program tree types of verification
conditions are generated: consistency, completeness and termination
conditions. They described invariant-based programming in SOCOS. In
\cite{back2011correct} this tool was used to verify heap sort
algorithm.

There are many tools for Java program developers maid to automatically
prove program correctness. Krakatoa Modeling Language (KML) is
described in \cite{spsa} with example of sorting
algorithms. Refinement is not supported in KML and any refinement
property could not automatically be proved. The language KML is also
not formally verified, but some parts are proved by Alt-Ergo, Simplify
and Yices. The paper proposed some improvements for working with
permutation and arrays in KML. Why/Krakatoa/Caduceus\cite{krakatoa} is
a tool for deductive program verification for Java and C. The approach
is to use Krakatoa and Caduceus to translate Java/C programs into Why
program. This language is suitable for program verification. The idea
is to generate verification conditions based on weakest precondition
calculus.


% ------------------------------------------------------------------------------
\section{Conclusions and Further Work}
\label{sec:conclusion}
% ------------------------------------------------------------------------------

In this paper we illustrated a proof management technology. The
methodology that we use in this paper for the formalization is
refinement: the formalization begins with a most basic specification,
which is then refined by introducing more advanced techniques, while
preserving the correctness. This incremental approach proves to be a
very natural approach in formalizing complex software systems. It
simplifies understanding of the system and reduces the overall
verification effort.

Modularity is very popular in nowadays imperative languages. This
approach could be used for software verification and Isabelle/HOL
locales provide means for modular reasoning. They support multiple
inheritance and this means that locales can imitate connections
between functions, procedures or objects. It is possible to establish
some general properties of an algorithm or to compare these
properties. So, it is possible to compare programs. And this is a
great advantage in program verification, something that is not done
very often. This could help in better understanding of an algorithm
which is essential for computer science education. So apart from being
able to formalize verification in easier manner, this approach gives
us opportunity to compare different programs. This was showed on
Selection and Heap sort example and the connection between these two
sorts was easy to comprehend. The value of this approach is not so
much in obtaining a nice implementation of some algorithm, but in
unraveling its structure. This is very important for computer science
education and this can help in better teaching and understanding of an
algorithms.

Using experience from this formalization, we came to conclusion that
the general principle for refinement in program verification should
be: {\em divide program into small modules (functions, classes) and
  verify each modulo separately in order that corresponds to the order
  in entire program implementation}. Someone may argue that this
principle was not followed in each step of formalization, for example
when we implemented {\em Selection sort} or when we defined {\em
  is\_heap} and {\em multiset} in one step, but we feel that those
function were simple and deviations in their implementations are
minimal.

The next step is to formally verify all sorting algorithms and using
refinement method to formally analyze and compare different sorting
algorithms.

% optional bibliography
\bibliographystyle{abbrv}
\bibliography{root}

\end{document}

%%% Local Variables:
%%% mode: latex
%%% TeX-master: t
%%% End:
