\documentclass[11pt,a4paper,fleqn]{article}
\usepackage{isabelle,isabellesym}
\renewcommand{\isastyletxt}{\isastyletext}

% further packages required for unusual symbols (see also
% isabellesym.sty), use only when needed

%\usepackage{amssymb}
  %for \<leadsto>, \<box>, \<diamond>, \<sqsupset>, \<mho>, \<Join>,
  %\<lhd>, \<lesssim>, \<greatersim>, \<lessapprox>, \<greaterapprox>,
  %\<triangleq>, \<yen>, \<lozenge>

%\usepackage{eurosym}
  %for \<euro>

%\usepackage[only,bigsqcap]{stmaryrd}
  %for \<Sqinter>

%\usepackage{eufrak}
  %for \<AA> ... \<ZZ>, \<aa> ... \<zz> (also included in amssymb)

%\usepackage{textcomp}
  %for \<onequarter>, \<onehalf>, \<threequarters>, \<degree>, \<cent>,
  %\<currency>

% this should be the last package used
\usepackage{pdfsetup}

% urls in roman style, theory text in math-similar italics
\urlstyle{rm}
\isabellestyle{it}

% for uniform font size
%\renewcommand{\isastyle}{\isastyleminor}


\begin{document}

\title{The Relational Method with Message Anonymity\\for the Verification of Cryptographic Protocols}
\author{Pasquale Noce\\Software Engineer at HID Global, Italy\\pasquale dot noce dot lavoro at gmail dot com\\pasquale dot noce at hidglobal dot com}
\maketitle

\begin{abstract}
This paper introduces a new method for the formal verification of cryptographic protocols, the
relational method, derived from Paulson's inductive method by means of some enhancements aimed at
streamlining formal definitions and proofs, specially for protocols using public key cryptography.
Moreover, this paper proposes a method to formalize a further security property, message anonymity,
in addition to message confidentiality and authenticity.

The relational method, including message anonymity, is then applied to the verification of a sample
authentication protocol, comprising Password Authenticated Connection Establishment (PACE) with Chip
Authentication Mapping followed by the explicit verification of an additional password over the PACE
secure channel.
\end{abstract}

\tableofcontents

% sane default for proof documents
\parindent 0pt\parskip 0.5ex

% generated text of all theories
\input{session}

% bibliography
\bibliographystyle{abbrv}
\bibliography{root}

\end{document}
