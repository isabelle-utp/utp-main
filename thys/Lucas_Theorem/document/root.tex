\documentclass[11pt,a4paper]{article}
\usepackage{isabelle,isabellesym}
\usepackage{amsmath}

% further packages required for unusual symbols (see also
% isabellesym.sty), use only when needed

%\usepackage{amssymb}
  %for \<leadsto>, \<box>, \<diamond>, \<sqsupset>, \<mho>, \<Join>,
  %\<lhd>, \<lesssim>, \<greatersim>, \<lessapprox>, \<greaterapprox>,
  %\<triangleq>, \<yen>, \<lozenge>

%\usepackage{eurosym}
  %for \<euro>

%\usepackage[only,bigsqcap]{stmaryrd}
  %for \<Sqinter>

%\usepackage{eufrak}
  %for \<AA> ... \<ZZ>, \<aa> ... \<zz> (also included in amssymb)

%\usepackage{textcomp}
  %for \<onequarter>, \<onehalf>, \<threequarters>, \<degree>, \<cent>,
  %\<currency>

% this should be the last package used
\usepackage{pdfsetup}

% urls in roman style, theory text in math-similar italics
\urlstyle{rm}
\isabellestyle{it}

% for uniform font size
%\renewcommand{\isastyle}{\isastyleminor}


\begin{document}

\title{Lucas's Theorem}
\author{Chelsea Edmonds}
\maketitle

\begin{abstract}
  This work presents a formalisation of a generating function proof for Lucas's theorem. We first outline extensions to the existing Formal Power Series (FPS) library, including an equivalence relation for coefficients modulo $n$, an alternate binomial theorem statement, and a formalised proof of the Freshman's dream (mod $p$) lemma.

  The second part of the work presents the formal proof of Lucas's Theorem. Working backwards, the formalisation first proves a well known corollary of the theorem which is easier to formalise and then applies induction to prove the original theorem statement. The proof of the corollary aims to provide a good example of a formalised generating function equivalence proof using the FPS library. The final theorem statement is intended to be integrated into the formalised proof of Hilbert's 10th Problem \cite{bayerDPRMTheoremIsabelle2019}.
\end{abstract}

\tableofcontents

% sane default for proof documents
\parindent 0pt\parskip 0.5ex

% generated text of all theories
\input{session}

% optional bibliography
\bibliographystyle{abbrv}
\bibliography{root}

\end{document}

%%% Local Variables:
%%% mode: latex
%%% TeX-master: t
%%% End:
