\documentclass[11pt,a4paper]{article}
\usepackage{isabelle,isabellesym}

\usepackage{amsmath,amssymb,amsthm}

% this should be the last package used
\usepackage{pdfsetup}

% urls in roman style, theory text in math-similar italics
\urlstyle{rm}
\isabellestyle{it}

\newtheoremstyle{normalfont}{ }{ }{\normalfont}{ }{\normalfont\bfseries}{}{ }{}
\theoremstyle{normalfont}
\newtheorem*{notation}{Notation}

\newcommand{\flush}{\leavevmode\newline\vspace*{-1.5em}}
\newcommand{\Ex}{\operatorname{E}}
\newcommand{\Var}{\operatorname{Var}}
\newenvironment{remark}{\begin{small}$\RHD$ Remark: \itshape}{\end{small}}

\begin{document}

\title{Properties of Random Graphs -- Subgraph Containment}
\author{Lars Hupel}
\maketitle

  \begin{abstract}
    Random graphs are graphs with a fixed number of vertices, where each edge is present with a fixed probability.
    We are interested in the probability that a random graph contains a certain pattern, for example a cycle or a clique.
    A very high edge probability gives rise to perhaps too many edges (which degrades performance for many algorithms), whereas a low edge probability might result in a disconnected graph.
    We prove a theorem about a threshold probability such that a higher edge probability will asymptotically almost surely produce a random graph with the desired subgraph.
  \end{abstract}

\tableofcontents

\newpage

% sane default for proof documents
\parindent 0pt\parskip 0.5ex

\section{Introduction}

Random graphs have been introduced by Erd\H{o}s and R\'enyi in \cite{erdos}.
They describe a probability space where, for a fixed number of vertices, each possible edge is present with a certain probability independent from other edges, but with the same probability for each edge.
They study what properties emerge when increasing the number of vertices, or as they call it, ``the evolution of such a random graph''.
The theorem which we will prove here is a slightly different version from that in the first section of that paper.

Here, we are interested in the probability that a random graph contains a certain pattern, for example a cycle or a clique.
A very high edge probability gives rise to perhaps too many edges, which is usually undesired since it degrades the performance of many algorithms, whereas a low edge probability might result in a disconnected graph.
The central theorem determines a threshold probability such that a higher edge probability will asymptotically almost surely produce a random graph with the desired subgraph.

The proof is outlined in \cite[\S\ 11.4]{graph-theory} and \cite[\S\ 3]{random-graphs}.
The work is based on the comprehensive formalization of probability theory in Isabelle/HOL and on a previous definition of graphs in a work by Noschinski \cite{girth-chromatic-afp}.
There, Noschinski formalized the proof that graphs with arbitrarily large girth and chromatic number exist.
While the proof in this paper uses a different approach, the definition of a probability space on edges turned out to be quite useful. 


% generated text of all theories
\input{session}

% optional bibliography
\bibliographystyle{abbrv}
\bibliography{root}

\end{document}

%%% Local Variables:
%%% mode: latex
%%% TeX-master: t
%%% End:
