\documentclass[11pt,a4paper]{article}
\usepackage{isabelle,isabellesym}

% further packages required for unusual symbols (see also
% isabellesym.sty), use only when needed

%\usepackage{amssymb}
  %for \<leadsto>, \<box>, \<diamond>, \<sqsupset>, \<mho>, \<Join>,
  %\<lhd>, \<lesssim>, \<greatersim>, \<lessapprox>, \<greaterapprox>,
  %\<triangleq>, \<yen>, \<lozenge>

%\usepackage{eurosym}
  %for \<euro>

%\usepackage[only,bigsqcap]{stmaryrd}
  %for \<Sqinter>

%\usepackage{eufrak}
  %for \<AA> ... \<ZZ>, \<aa> ... \<zz> (also included in amssymb)

%\usepackage{textcomp}
  %for \<onequarter>, \<onehalf>, \<threequarters>, \<degree>, \<cent>,
  %\<currency>

% this should be the last package used
\usepackage{pdfsetup}

% urls in roman style, theory text in math-similar italics
\urlstyle{rm}
\isabellestyle{it}

% for uniform font size
%\renewcommand{\isastyle}{\isastyleminor}


\begin{document}

\title{$\Sigma$-protocols and Commitment Schemes}
\author{David Butler, Andreas Lochbihler}
\maketitle

\begin{abstract}
	We use CryptHOL~\cite{Basin2017} to formalise commitment schemes and $\Sigma$-protocols. Both are widely used fundamental two party cryptographic primitives. Security for commitment schemes is considered using game-based definitions whereas the security of $\Sigma$-protocols is considered using both the game-based and simulation-based security paradigms. In this work we first define security for both primitives and then prove secure multiple examples namely; the Schnorr, Chaum-Pedersen and Okamoto $\Sigma$-protocols as well as a construction that allows for compound (AND and OR) $\Sigma$-protocols  and the Pedersen and Rivest commitment schemes. We also prove that commitment schemes can be constructed from $\Sigma$-protocols. We formalise this proof at an abstract level, only assuming the existence of a $\Sigma$-protocol, consequently the instantiations of this result for the concrete $\Sigma$-protocols we consider come for free.
	
\end{abstract}

\tableofcontents

% sane default for proof documents
\parindent 0pt\parskip 0.5ex

% generated text of all theories
\input{session}

% optional bibliography
\bibliographystyle{abbrv}
\bibliography{root}

\end{document}

%%% Local Variables:
%%% mode: latex
%%% TeX-master: t
%%% End:
