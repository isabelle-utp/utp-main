\documentclass[11pt,DIV10,a4paper,twoside=semi,openright,titlepage]{scrreprt}
\usepackage{fixltx2e}
%%%%%%%%%%%%%%%%%%%%%%%%%%%%%%%%%%%%%%%%%%%%%%%%%%%%%%%%%%%%%%%%%%%%%%%
%%% Overrides the (rightfully issued) warning by Koma Script that \rm
%%% etc. should not be used (they are deprecated since more than a
%%% decade)
  \DeclareOldFontCommand{\rm}{\normalfont\rmfamily}{\mathrm}
  \DeclareOldFontCommand{\sf}{\normalfont\sffamily}{\mathsf}
  \DeclareOldFontCommand{\tt}{\normalfont\ttfamily}{\mathtt}
  \DeclareOldFontCommand{\bf}{\normalfont\bfseries}{\mathbf}
  \DeclareOldFontCommand{\it}{\normalfont\itshape}{\mathit}
%%%%%%%%%%%%%%%%%%%%%%%%%%%%%%%%%%%%%%%%%%%%%%%%%%%%%%%%%%%%%%%%%%%%%%%

\usepackage{isabelle,isabellesym}
\usepackage{stmaryrd}
\usepackage{paralist}
\usepackage{xspace}
\newcommand{\testgen}{HOL-TestGen\xspace}
\newcommand{\testgenFW}{HOL-TestGen/FW\xspace}
\usepackage[numbers, sort&compress, sectionbib]{natbib}
\usepackage{graphicx}
\usepackage{color}
\sloppy

\usepackage{amssymb}



\newcommand{\isasymmodels}{\isamath{\models}}
\newcommand{\HOL}{HOL}

\newcommand{\ie}{i.\,e.}
\newcommand{\eg}{e.\,g.}

\usepackage{pdfsetup}

\urlstyle{rm}
\isabellestyle{it}
\renewcommand{\isastyle}{\isastyleminor}

\pagestyle{empty} 
\begin{document}
\renewcommand{\subsubsectionautorefname}{Section}
\renewcommand{\subsectionautorefname}{Section}
\renewcommand{\sectionautorefname}{Section}
\renewcommand{\chapterautorefname}{Chapter}
\newcommand{\subtableautorefname}{\tableautorefname}
\newcommand{\subfigureautorefname}{\figureautorefname}

\title{The Unified Policy Framework\\
  (UPF)}
\author{Achim D. Brucker\footnotemark[1] \quad
        Lukas Br\"ugger\footnotemark[2]  \quad
        Burkhart Wolff\footnotemark[3]\\[1.5em]
  \normalsize
  \normalsize\footnotemark[1]~SAP SE, Vincenz-Priessnitz-Str. 1, 76131 Karlsruhe,
  Germany \texorpdfstring{\\}{}
  \normalsize\href{mailto:"Achim D. Brucker"
    <achim.brucker@sap.com>}{achim.brucker@sap.com}\\[1em]
  %
  \normalsize\footnotemark[2]Information Security, ETH Zurich, 8092 Zurich, Switzerland
  \texorpdfstring{\\}{}
  \normalsize\href{mailto:"Lukas Bruegger"
    <lukas.a.bruegger@gmail.com>}{Lukas.A.Bruegger@gmail.com}\\[1em]
  %
  \normalsize\footnotemark[3]~Univ. Paris-Sud, Laboratoire LRI,
  UMR8623, 91405 Orsay, France
  France\texorpdfstring{\\}{}
  \normalsize\href{mailto:"Burkhart Wolff" <burkhart.wolff@lri.fr>}{burkhart.wolff@lri.fr}
}

\pagestyle{empty}
\publishers{%
  \normalfont\normalsize%
    \centerline{\textsf{\textbf{\large Abstract}}}
    \vspace{1ex}%
    \parbox{0.8\linewidth}{%
      We present the \emph{Unified Policy Framework}
      (UPF), a generic framework for modelling security
      (access-control) policies; in Isabelle/\HOL. 
      %\cite{}.
      UPF emphasizes the view that a policy is a policy decision
      function that grants or denies access to resources, permissions,
      etc. In other words, instead of modelling the
      relations of permitted or prohibited requests directly, we model
      the concrete function that implements the policy decision point
      in a system, seen as an ``aspect'' of ``wrapper'' around the 
      business logic  % Fachlogik 
      of a system.
      In more detail, UPF is based on the following four  principles:
      \begin{inparaenum}
       \item Functional representation of policies,
       \item No conflicts are possible,
       \item Three-valued decision type (allow, deny, undefined),
       \item Output type not containing the decision only.
      \end{inparaenum}
   }
}

\maketitle
\cleardoublepage
\pagestyle{plain}
\tableofcontents
\cleardoublepage

\section{Introduction}

This document is based on
\cite{ArkoudasETAL04VerifyingFileSystemImplementationICFEM}, which
explores the challenges of verifying the core operations of a
Unix-like file system \cite{thompson78unix,mckusick84fast}.  The paper
\cite{ArkoudasETAL04VerifyingFileSystemImplementationICFEM} formalizes
the specification of the file system as a map from file names to
sequences of bytes, then formalizes an implementation that uses such
standard file system data structures as i-nodes and fixed-sized disk
blocks.  The correctness of the
implementation is verified by proving the existence of a simulation relation
\cite{RoeverEngelhardt98DataRefinement} between the specification and
the implementation.  The original effort of
\cite{ArkoudasETAL04VerifyingFileSystemImplementationICFEM} started in
Isabelle.  The process of developing the proof in Isabelle helped to 
remove the initial bugs in the concrete and
abstract models (though the proof has not been completed so far).  

Here we present a completed proof for a simplified problem:
data refinement of a single file.  We present operations on
both abstract and concrete files, define a function mapping
concrete files to abstract files, and prove that this
function is a simulation relation.

We use two libraries of arrays: arrays without bounds
checks, which can be thought of as an array with an
unbounded number of elements, and resizable arrays, which
have a notion of the current size, but expand in response to
array writes that are outside the current bounds.

%%%%%%%%%%%%%%%%%%%%%%%%%%%%%%
% <session>
  % \input{session}
  \chapter{The Unified Policy Framework (UPF)}
  \input{UPFCore}
  \input{ElementaryPolicies}
  \input{SeqComposition}
  \input{ParallelComposition}
  \input{Analysis}
  \input{Normalisation}
  \input{NormalisationTestSpecification}
  \input{UPF}
  \chapter{Example}
  In this chapter, we present a small example application of UPF for
modeling access control for a Web service that might be used in a
hospital. This scenario is motivated by our formalization of the NHS
system~\cite{bruegger:generation:2012,brucker.ea:model-based:2011}. 

UPF was also successfully used for modeling network security policies
such as the ones enforced by
firewalls~\cite{bruegger:generation:2012,brucker.ea:formal-fw-testing:2014}. These
models were also used for generating test cases using
HOL-TestGen~\cite{brucker.ea:theorem-prover:2012}.

  \input{Service}
  \input{ServiceExample}
% </session>
%%%%%%%%%%%%%%%%%%%%%%%%%%%%%%
\section{Conclusion}\label{sec:concl}
  We have presented a verification of two variants of Gabow's algorithm: Computation of the strongly connected components of
  a graph, and emptiness check of a generalized B\"uchi automaton. We have extracted efficient code with a performance comparable to a
  reference implementation in Java.
  
  We have modularized the formalization in two directions: First, we share most of the proofs between the two variants of the algorithm. Second,
  we use a stepwise refinement approach to separate the algorithmic ideas and the correctness proof from implementation details.
  Sharing of the proofs reduced the overall effort of developing both algorithms. Using a stepwise refinement approach allowed us to
  formalize an efficient implementation, without making the correctness proof complex and unmanageable by cluttering it with implementation details.

  Our development approach is independent of Gabow's algorithm, and can be re-used for the verification of other algorithms.

  \paragraph{Current and Future Work} 
  An important direction of future work is to fine-tune the implementation of 
  the emptiness check algorithm for speed, as speed of the checking algorithm
  directely influences the performance of the modelchecker.

  
\chapter{Appendix}
\input{Monads}

%%% Local Variables:
%%% mode: latex
%%% TeX-master: "root"
%%% End:


\nocite{brucker.ea:formal-fw-testing:2014,brucker.ea:hol-testgen-fw:2013,brucker.ea:theorem-prover:2012,brucker.ea:model-based:2011}
\nocite{bruegger:generation:2012}
\bibliographystyle{abbrvnat}
\bibliography{root}


\end{document}

%%% Local Variables:
%%% mode: latex
%%% TeX-master: t
%%% End:
