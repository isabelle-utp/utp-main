\documentclass[11pt,a4paper]{article}
\usepackage{isabelle,isabellesym}
\usepackage{fullpage}
\usepackage[usenames,dvipsnames]{color}
\usepackage{document}

% further packages required for unusual symbols (see also
% isabellesym.sty), use only when needed

\usepackage{amssymb}
  %for \<leadsto>, \<box>, \<diamond>, \<sqsupset>, \<mho>, \<Join>,
  %\<lhd>, \<lesssim>, \<greatersim>, \<lessapprox>, \<greaterapprox>,
  %\<triangleq>, \<yen>, \<lozenge>

\usepackage[english]{babel}
  %option greek for \<euro>
  %option english (default language) for \<guillemotleft>, \<guillemotright>

\usepackage{stmaryrd}
  %for \<Sqinter>

\usepackage{eufrak}
  %for \<AA> ... \<ZZ>, \<aa> ... \<zz> (also included in amssymb)

%\usepackage{textcomp}
  %for \<onequarter>, \<onehalf>, \<threequarters>, \<degree>, \<cent>,
  %\<currency>

% this should be the last package used
\usepackage{pdfsetup}

% urls in roman style, theory text in math-similar italics
\urlstyle{rm}
\isabellestyle{it}

% for uniform font size
%\renewcommand{\isastyle}{\isastyleminor}

\begin{document}

\title{Theory of Designs in Isabelle/UTP}

\author{Simon Foster \and Yakoub Nemouchi \and Frank Zeyda}

\maketitle

\begin{abstract}
  This document describes a mechanisation of the UTP theory of designs in Isabelle/UTP. Designs enrich UTP relations
  with explicit precondition/postcondition pairs, as present in formal notations like VDM, B, and the refinement
  calculus. If a program's precondition holds, then it is guaranteed to terminate and establish its postcondition, which
  is an approach known as total correctness. If the precondition does not hold, the behaviour is maximally nondeterministic,
  which represents unspecified behaviour. In this mechanisation, we create the theory of designs, including its
  alphabet, signature, and healthiness conditions. We then use these to prove the key algebraic laws of
  programming. This development can be used to support program verification based on total correctness.
\end{abstract}

\tableofcontents

% sane default for proof documents
\parindent 0pt\parskip 0.5ex

% generated text of all theories
\input{session}

% optional bibliography
\bibliographystyle{abbrv}
\bibliography{root}

\end{document}
